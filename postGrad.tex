
% Save this as tutorial.tex for the lwarp package tutorial.

% Listings works with book
\documentclass{book}

\usepackage{iftex}

% --- LOAD FONT SELECTION AND ENCODING BEFORE LOADING LWARP ---

\ifPDFTeX
\usepackage{lmodern}            % pdflatex or dvi latex
\usepackage[T1]{fontenc}
\usepackage[utf8]{inputenc}
\else
\usepackage{fontspec}           % XeLaTeX or LuaLaTeX
\fi

% --- LWARP IS LOADED NEXT ---
\usepackage[
HomeHTMLFilename=index,     % Filename of the homepage.
%   HTMLFilename={node-},       % Filename prefix of other pages.
%   IndexLanguage=english,      % Language for xindy index, glossary.
%   latexmk,                    % Use latexmk to compile.
%   OSWindows,                  % Force Windows. (Usually automatic.)
% Mathjax is the way to go, if I lose mathjax, host it myself????? Don't have too much math involved in my equations.
mathjax,                    % Use MathJax to display math.
]{lwarp}
% \boolfalse{FileSectionNames}  % If false, numbers the files.


% --- Custom non tutorial content
% --- Listings
\usepackage{listings}

% Styling only for pdf --- for now
\lstset{language=C++,
	basicstyle=\ttfamily,
	keywordstyle=\color{blue}\ttfamily,
	stringstyle=\color{red}\ttfamily,
	commentstyle=\color{green}\ttfamily,
	morecomment=[l][\color{magenta}]{\#},
	captionpos=t,
}

% --- Glossary Terms
\usepackage[acronym, toc, xindy]{glossaries}
% Hands on Systems Programming C++
\newglossaryentry{api}
{
   name={API},
   description={An Application Programming Interface (API) is a particular set
           of rules and specifications that a software program can follow to access and make use of the services and resources provided by another particular software program that implements that API},
   first={Application Programming Interface (API)},
   long={Application Programming Interface}
}

\newglossaryentry{staticlib}
{
	name={Static libraries},
	description ={
			libraries that are linked at compile time
	}
}

\newglossaryentry{dynamiclib}
{
	name={Dynamic libraries},
	description ={
			libraries that are linked at run time
	}
}

%%% Laravel Notes

\newglossaryentry{depInj}
{
	name={Dependency Injection},
	description={
		Dependency injection means that, rather than being instantiated ("newed up") within a class, each class's dependencies will be injected in from the outside.
	}
}

\newglossaryentry{IoC}
{
	name={IoC},
	description={
		In software engineering, inversion of control (IoC) is a programming principle. IoC inverts the flow control as compared to traditional control flow. In IoC, custom-written portions of a computer program receive the flow of control from a generic framework. A software architecture with this design inverts control as compared to traditional procedural programming: in traditional programming, the custom code that expresses the purpose of the program calls into reusable libraries to take care of generic tasks, but with inversion of control, it is the framework that calls into the custom, or task-specific, code.
	},
	first={Inversion of Control},
	long={Inversion of Control}
}

%% --- GAN Notes ---
\newglossaryentry{GAN}
{
	name={GAN},
	description={
		A GAN is a deep neural network architecture made up of two networks, a generator network and a discriminator network. Through multiple cycles of generation and discrimination, both networks train each other, while simultaneously trying to outwit each other.
	},
	first={Generative Adversarial Networks},
	long={Generative Adversarial Networks}
}
% --- Software Tools ---
\newglossaryentry{git}
{
	name={git},
	description ={
			the most popular version control system used for tracking changes in source code and coordinating work on those files among multiple people
	}
}

% --- Quantum Computing ---


% --- Algorithms ---
\newglossaryentry{Adapter}
{
	name = {Adapter},
	description = {
		The adapter pattern provides a wrapper with an interface
		required by the \gls{api} client to link incompatible types and act as a
		translator between the two types. The adapter uses the interface
		of a class to be a class with another compatible interface. When
		requirements change, there are scenarios where class
		functionality needs to be changed because of incompatible
		interfaces.
	}
}

\newglossaryentry{Bridge}
{
	name={Bridge},
	description={
			Bridge decouples the implementation from the abstraction. The
		abstract base class can be subclassed to provide different
		implementations and allow implementation details to be
		modified easily. 
	}
}

 % Glossary File
\makeglossaries

% Hands on Systems Programming C++
\newacronym{api}{API}{application programming interface}

\newacronym{abi}{ABI}{application binary interface}

\newacronym{qpu}{QPU}{Quantum Processing Unit}
\glsaddall


% --- MDFFRAMED CUSTOMIZATION
\usepackage{mdframed}

\mdfdefinestyle{example}{
	linecolor=blue!50,
	align=center, 
	backgroundcolor=blue!30, 
	frametitlebackgroundcolor=blue!15,
	shadow=true
}
\mdfdefinestyle{definition}{
	linecolor=red!50,
	align=center, 
	backgroundcolor=red!30, 
	frametitlebackgroundcolor=red!15,
	shadow=true
}
\mdfdefinestyle{theorem}{
	linecolor=yellow!50,
	align=center, 
	backgroundcolor=yellow!30, 
	frametitlebackgroundcolor=yellow!15,
	shadow=true
}

\mdfdefinestyle{important}{
	linecolor=purple!50,
	align=center, 
	backgroundcolor=purple!30, 
	frametitlebackgroundcolor=purple!15,
	shadow=true
}

% --- References ---
\newcommand{\onlineCite}{[Online] Available: }	% Used in BIBLIOGRAPHY 
\usepackage[backend=bibtex,sorting=none]{biblatex}	% Sort by citation order, IEEE
\addbibresource{config/references.bib}

% --- MATH LIBRARIES ---
\usepackage{amssymb}
% --- LOAD PDFLATEX MATH FONTS HERE ---

% --- OTHER PACKAGES ARE LOADED AFTER LWARP ---
\usepackage{makeidx} \makeindex
\usepackage{xcolor}             % (Demonstration purposes only.)
\usepackage{hyperref,cleveref}  % LOAD THESE LAST!

% --- LATEX AND HTML CUSTOMIZATION ---
\title{The Lwarp Tutorial}
\author{Some Author}
\setcounter{tocdepth}{2}        % Include subsections in the \TOC.
\setcounter{secnumdepth}{2}     % Number down to subsections.
\setcounter{FileDepth}{1}       % Split \HTML\ files at sections
\booltrue{CombineHigherDepths}  % Combine parts/chapters/sections
\setcounter{SideTOCDepth}{1}    % Include subsections in the side\TOC
\HTMLTitle{Webpage Title}       % Overrides \title for the web page.
\HTMLAuthor{Some Author}        % Sets the HTML meta author tag.
\HTMLLanguage{en-US}            % Sets the HTML meta language.
\HTMLDescription{A description.}% Sets the HTML meta description.
\HTMLFirstPageTop{Name and \fbox{HOMEPAGE LOGO}}
\HTMLPageTop{\fbox{LOGO}}
\HTMLPageBottom{Contact Information and Copyright}
\CSSFilename{lwarp_sagebrush.css}

%%%%%%%%%%%%%%%%%%%%%%%%%%%%%%%%%%%%%%%%%%%%%%%%%%%%%%%%%%%%%%%%%%%%%%%%%%%%%%%%%
%%%%%%%%%%%%%%%%%%%%%%%%%%%% START OF PANDOC %%%%%%%%%%%%%%%%%%%%%%%%%%%%%%%%%%%%
%%%%%%%%%%%%%%%%%%%%%%%%%%%%%%%%%%%%%%%%%%%%%%%%%%%%%%%%%%%%%%%%%%%%%%%%%%%%%%%%%
\usepackage{booktabs}
\usepackage{fancyvrb}
\DefineShortVerb[commandchars=\\\{\}]{\|}
% Add ',fontsize=\small' for more characters per line
\newenvironment{Shaded}{}{}
\newcommand{\KeywordTok}[1]{\textcolor[rgb]{0.00,0.44,0.13}{\textbf{{#1}}}}
\newcommand{\DataTypeTok}[1]{\textcolor[rgb]{0.56,0.13,0.00}{{#1}}}
\newcommand{\DecValTok}[1]{\textcolor[rgb]{0.25,0.63,0.44}{{#1}}}
\newcommand{\BaseNTok}[1]{\textcolor[rgb]{0.25,0.63,0.44}{{#1}}}
\newcommand{\FloatTok}[1]{\textcolor[rgb]{0.25,0.63,0.44}{{#1}}}
\newcommand{\CharTok}[1]{\textcolor[rgb]{0.25,0.44,0.63}{{#1}}}
\newcommand{\StringTok}[1]{\textcolor[rgb]{0.25,0.44,0.63}{{#1}}}
\newcommand{\CommentTok}[1]{\textcolor[rgb]{0.38,0.63,0.69}{\textit{{#1}}}}
\newcommand{\OtherTok}[1]{\textcolor[rgb]{0.00,0.44,0.13}{{#1}}}
\newcommand{\AlertTok}[1]{\textcolor[rgb]{1.00,0.00,0.00}{\textbf{{#1}}}}
\newcommand{\FunctionTok}[1]{\textcolor[rgb]{0.02,0.16,0.49}{{#1}}}
\newcommand{\RegionMarkerTok}[1]{{#1}}
\newcommand{\ErrorTok}[1]{\textcolor[rgb]{1.00,0.00,0.00}{\textbf{{#1}}}}
\newcommand{\NormalTok}[1]{{#1}}
\setlength{\parindent}{0pt}
\setlength{\parskip}{6pt plus 2pt minus 1pt}
\setlength{\emergencystretch}{3em}  % prevent overfull lines
\setcounter{secnumdepth}{0}


%%%%%%%%%%%%%%%%%%%%%%%%%%%%%%%%%%%%%%%%%%%%%%%%%%%%%%%%%%%%%%%%%%%%%%%%%%%%%%%%%
%%%%%%%%%%%%%%%%%%%%%%%%%%%%%% END OF PANDOC %%%%%%%%%%%%%%%%%%%%%%%%%%%%%%%%%%%%
%%%%%%%%%%%%%%%%%%%%%%%%%%%%%%%%%%%%%%%%%%%%%%%%%%%%%%%%%%%%%%%%%%%%%%%%%%%%%%%%%

%%%%%%%%%%%%%%%%%%%%%%%%%%%%%%%%%%%%%%%%%%%%%%%%%%%%%%%%%%%%%%%%%%%%%%%%%%%%%%%%
%%%%%%%%%%%%%%%%%%%%%%%% UTILITY COMMANDS %%%%%%%%%%%%%%%%%%%%%%%%%%%%%%%%%%%%%%
%%%%%%%%%%%%%%%%%%%%%%%%%%%%%%%%%%%%%%%%%%%%%%%%%%%%%%%%%%%%%%%%%%%%%%%%%%%%%%%%

%%%%%%%%%%%%%%%%%%%%%%%%%%%%%%%%%%%%%%%%%%%%%%%%%%%%%%%%%%%%%%%%%%%%%%%%%%%%%%%%
%%%%%%%%%%%%%%%%%%%%% END UTILITY COMMANDS %%%%%%%%%%%%%%%%%%%%%%%%%%%%%%%%%%%%%
%%%%%%%%%%%%%%%%%%%%%%%%%%%%%%%%%%%%%%%%%%%%%%%%%%%%%%%%%%%%%%%%%%%%%%%%%%%%%%%%

%%%%%%%%%%%%%%%%%%%%%%%%%%%%%%%%%%%%%%%%%%%%%%%%%%%%%%%%%%%%%%%%%%%%%%%%%%%%%%%%%
%%%%%%%%%%%%%%%%%%%%%%%%%%%%%% REFERENCE COMMANDS OF INTEREST %%%%%%%%%%%%%%%%%%%
%%%%%%%%%%%%%%%%%%%%%%%%%%%%%%%%%%%%%%%%%%%%%%%%%%%%%%%%%%%%%%%%%%%%%%%%%%%%%%%%%

%  replacementTerms = ["Cpp Code", cplusplus, "Latex Code", latex, "Python Script", python, 
% "Bash Script", bash,"Matlab Script", matlab,"Yaml File",yaml,"JSON Output", json, "Golang", golang]
% These are the terms that are manually converted to code listings, by default it is converted to cpp, thanks to SENG475

% Other Notes

% --- Mdframed examples
% 	Values include:	example, definition, theorem and important
%
%	\begin{mdframed}[style=definition, frametitle={SENG475 Example Code Listing}]
%		All for the sake of glory city. We see that in \cref{ex:1} that mdframed makes 
%		environments look nice
%	\end{mdframed}
%	
%	\begin{mdframed}[style=theorem, frametitle={SENG475 Example Code Listing}]
%		All for the sake of glory city. We see that in \cref{ex:1} that mdframed makes 
%		environments look nice
%	\end{mdframed}
%	
%	\begin{mdframed}[style=important, frametitle={SENG475 Example Code Listing}]
%		All for the sake of glory city. We see that in \cref{ex:1} that mdframed makes 
%		environments look nice
%	\end{mdframed}
\begin{document}


\begin{warpHTML}
	\maketitle                   % Or titlepage/titlingpage environment.
\end{warpHTML}
\begin{center}
	%		{\vspace*{3pt} }
	{\Huge \textsc{Personal Study Notes} \\ \vspace{40pt}}
	%	{\Huge  \\ \vspace{4pt}}  
	\rule[13pt]{1\textwidth}{1pt} \\ \vspace{1pt}
	{\LARGE \textbf{{\textsc{Software Engineer}}} \\ }
	{\Large \textsc{Post Grad Notes} \\} 
	\vspace{4pt} 
	%{\Large \textsc{Spring 2018}} \\ 
	\vspace{20pt}
	{\Large \textsc{Victoria, British Columbia, Canada} \\ \vspace{45pt} }
	
	%		\\ \vspace{10pt}
	{\Large \textsc{\today} \\ \vspace{15pt}
		\textbf{Name:} \hfill David Li \\
		\textbf{Student Number:} \hfill V00818631 \\
		\textbf{Email:} \hfill \href{mailto:davidli012345@gmail.ca}{davidli012345@gmail.ca} \\
		\vspace{15pt}
		{\Large \textsc{In partial fulfillment of the never ending quest to learn. \\
			}
		}	
	}
\end{center}

% An article abstract would go here.

\tableofcontents                % MUST BE BEFORE THE FIRST SECTION BREAK!
\listoffigures
\lstlistoflistings % THIS SEEMS TO WORK WITH THE BOOK CLASS

\chapter{Example chapter}

\section{A section}

This is some text which is indexed.\index{Some text.}

\subsection{A subsection}

See \cref{fig:withtext}.

\begin{mdframed}[style=definition, frametitle={SENG475 Example Code Listing}]
	All for the sake of glory city. We see that in \cref{ex:1} that mdframed makes 
	environments look nice
\end{mdframed}

\begin{mdframed}[style=theorem, frametitle={SENG475 Example Code Listing}]
	All for the sake of glory city. We see that in \cref{ex:1} that mdframed makes 
	environments look nice
\end{mdframed}

\begin{mdframed}[style=important, frametitle={SENG475 Example Code Listing}]
	All for the sake of glory city. We see that in \cref{ex:1} that mdframed makes 
	environments look nice
\end{mdframed}
	
\begin{figure}\begin{center}
\fbox{\textcolor{blue!50!green}{Text in a figure.}}
\caption{A figure with text\label{fig:withtext}}
\end{center}\end{figure}

\begin{lstlisting}[language=C++, caption={Cpp Testing}]
#include <iostream>
\end{lstlisting}

% Note to self, refer to all go programs are go programs, golang is reserved for the code.
\begin{lstlisting}[caption={Golang Program}]
package main

import (
"fmt"
"io/ioutil"
)
\end{lstlisting}


\lstinputlisting[language=Octave, caption=Python Script]{scripts/code.py}

Using references is \cite{book:2300108}
\section{Some math}

Inline math: $r = r_0 + vt - \frac{1}{2}at^2$
followed by display math:
\begin{equation}
a^2 + b^2 = c^2
\end{equation}





\part{2020}


In Flutter, a full compile
generally takes less than 30 seconds, and incremental compiles take less than a second
thanks to hot reloading. 

At a high level, Flutter is a reactive, declarative, and composable view-layer library,
much like ReactJS on the web (but more like React mixed with the browser, because
Flutter is a complete rendering engine as well)


\section{Setting up Rasa on GCP}

Before running all commands switch to root user, ensure you have the latest version of pip
make a bare bones server in gcp micro works fine us-west-1, debian-10-buster

\begin{lstlisting}[language=C++, caption={Bash Script}]
sudo su
sudo apt install python3-pip
sudo -H pip3 install --upgrade pip
pip install rasa
rasa init
\end{lstlisting}

Make a repo called rasa bot.

Google Cloud platform -

Clear Storage and Clear Container Registry periodically each month to avoid paying 10 cents.

\chapter{spaCy Tutorials}

\begingroup
% For stupid UTF8 issues
\UseRawInputEncoding
\section{Spacy Code Tutorials}

This contains all the spacy code tutorial I have done and gone through.

\lstinputlisting[caption={spaCy Basic English NLP}]{chapters/2020/spacy/ch1/english.py}

\section*{Finding words, phrases, names and concepts}
\subsection{Document, spans and tokens}

\lstinputlisting[caption={spaCy First Token}]{chapters/2020/spacy/ch1/first_token.py}

\lstinputlisting[caption={spaCy Slice Doc}]{chapters/2020/spacy/ch1/slice_doc.py}

\subsection{Lexical attributes}

\lstinputlisting[caption={spaCy Detect Percentages}]{chapters/2020/spacy/ch1/get_percentage.py}

\subsection{Loading Models}


\lstinputlisting[caption={spaCy Loading Models}]{chapters/2020/spacy/ch1/loading_model.py}

\subsection{Linguistic Annotations}

\subsubsection{Thoughts}
This will be useful for me to grab the percentages from finance youtube videos.

\lstinputlisting[caption={spaCy Linguistic Annotations V1}]{chapters/2020/spacy/ch1/linguistic_annotations_v1.py}

\lstinputlisting[caption={spaCy Linguistic Annotations V2}]{chapters/2020/spacy/ch1/linguistic_annotations_v2.py}

\subsection{Named Entities}
\lstinputlisting[caption={spaCy Linguistic Annotations V2}]{chapters/2020/spacy/ch1/named_entities.py}

\subsection{Matcher}

\lstinputlisting[caption={spaCy Matcher}]{chapters/2020/spacy/ch1/matcher.py}

\lstinputlisting[caption={spaCy Writing Matcher}]{chapters/2020/spacy/ch1/writing_match_patterns1.py}

\lstinputlisting[caption={spaCy Writing Matcher II}]{chapters/2020/spacy/ch1/writing_match_patterns2.py}

\lstinputlisting[caption={spaCy Writing Matcher III}]{chapters/2020/spacy/ch1/writing_match_patterns3.py}

\section*{Large-scale data analysis with spaCy}

\lstinputlisting[caption={spaCy Cat hash}]{chapters/2020/spacy/ch2/cat_hash.py}

\lstinputlisting[caption={spaCy Person hash}]{chapters/2020/spacy/ch2/person_hash.py}

\lstinputlisting[caption={spaCy Docs, spans and entities from scratch}]{chapters/2020/spacy/ch2/entities.py}

\lstinputlisting[caption={spaCy Code rewrite to use spaCy functions}]{chapters/2020/spacy/ch2/code_writer.py}

\lstinputlisting[caption={spaCy Word Vectors}]{chapters/2020/spacy/ch2/word_vectors.py}

\lstinputlisting[caption={spaCy Similarities I}]{chapters/2020/spacy/ch2/similarities1.py}

\lstinputlisting[caption={spaCy Similarities II}]{chapters/2020/spacy/ch2/similarities2.py}

\lstinputlisting[caption={spaCy Extracting Countries and Relationship}]{chapters/2020/spacy/ch2/debugging_pattern.py}

\lstinputlisting[caption={spaCy Extracting Countries and Relationship}]{chapters/2020/spacy/ch2/extracting_countries_and_relationships.py}

\lstinputlisting[caption={spaCy Efficient Phrase}]{chapters/2020/spacy/ch2/efficient_phrase.py}

\endgroup

\textbf{The Problems with Immutability} 
Unfortunately, immutability is counter to how computers actually work. A machine has a
limited amount of memory.


The fallacies are these:

\begin{itemize}
\item The network is reliable.
\item Latency is zero.
\item Bandwidth is infinite.
\item The network is secure.
\item Topology doesn’t change.
\item There is one administrator.
\item Transport cost is zero.
\item The network is homogeneous.
\end{itemize}

Didn't feel like reading the rest.


% https://github.com/GoogleCloudPlatform/training-data-analyst/tree/master/courses/ai-for-finance


\section{GCP Finance course}

Took this course from finance
\href{https://github.com/GoogleCloudPlatform/training-data-analyst/tree/master/courses/ai-for-finance}{ai-for-finance}


Modern approaches use Machine Learning
to model fluid market behaviours and complexity.

\begin{itemize}
\item Financial markets are a dynamic, evolving collection of behaviours
\item Use historical data to train a model and adjusts
features and loadings to improve predictive power
\item Integrate the model into an order-execution strategy
\item Retrain and retest continually with new data
to capture the market's current state.
\end{itemize}


Market moves with many human factors that are difficult to measure.

% make into a note
Exogenous and endogenous factors both drive
changes in share prices. Technical strategies tend
to focus on endogenous factors and event-driven
strategies focus on exogenous factors.

% 04 04_what-do-we-want-to-model-lets-start-simple.mp4


\chapter{Real Content}

\begin{table}
\begin{tabular}{p{3cm} c c c c}Name & Category & Priority \\ \hline  examin pyalgotrade for stock selling and buying &  low &  investing \\ \hline  work on web scrapper experiment felgo &  webscrap &  high \\ \hline  gas station network  &  high &  dapps \\ \hline  explore ipfs solutions such as pinata and textile &   &  \\ \hline  add dash auth to dashboard &  finance &  high \\ \hline
\end{tabular}
\caption{\textbf{Todo List 2019/8/5}}
\end{table}

Enable Google Cloud Build, need to update google cloud sdk on windows, might broken.

Also it seems that google cloud build.




\begin{mdframed}[style=important, frametitle={QPU Versus GPU: Some Common Characteristics}]

\begin{itemize}
\item It is very rare that a program will run entirely on a QPU. Usually, a program running on a CPU will issue QPU instructions, and later retrieve the results.
\item Some tasks are very well suited to the QPU, and others are not.
\item The QPU runs on a separate clock from the CPU, and usually has its own dedicated hardware interfaces to external devices (such as optical outputs).
\item A typical QPU has its own special RAM, which the CPU cannot efficiently access.
\item Here's a list of pertinent facts about what it’s like to program a QPU:
It is very rare that a program will run entirely on a QPU. Usually, a program running on a CPU will issue QPU instructions, and later retrieve the results.
Some tasks are very well suited to the QPU, and others are not.
The QPU runs on a separate clock from the CPU, and usually has its own dedicated hardware interfaces to external devices (such as optical outputs).
A typical QPU has its own special RAM, which the CPU cannot efficiently access.
\end{itemize}
\end{mdframed}

% Having java here, doesn't matter because the language gets mapped in the html anyway using prism
\begin{lstlisting}[language=java, caption=Javascript Program: Quantum random spy hunter]
Example 2-4. Quantum random spy hunter
qc.reset(3);
qc.discard();
var a = qint.new(1, 'alice');
var fiber = qint.new(1, 'fiber');
var b = qint.new(1, 'bob');

function random_bit(q) {
    q.write(0);
    q.had();
    return q.read();
}

// Generate two random bits
var send_had = random_bit(a);
var send_val = random_bit(a);

// Prepare Alice's qubit
a.write(0);
if (send_val)  // Use a random bit to set the value
    a.not();
if (send_had)  // Use a random bit to apply HAD or not
    a.had();

// Send the qubit!
fiber.exchange(a);

// Activate the spy
var spy_is_present = true;
\end{lstlisting}

\textbf{Magneta Music Generation} Some of the beats generated were acceptable, could make sound effects from randomized shorter effects.

Probably just trial and error as for images, swipe them off the internet, because I still need animations for now.


\section{SQL}

Query
\begin{lstlisting}[caption={SQL Query for length}]
SELECT first_name, length(first_name) as first_length, last_name, length(last_name) as last_length, length(first_name)+length(last_name) as total_length FROM users WHERE length(first_name)+length(last_name) > 30;
\end{lstlisting}

\begin{lstlisting}[caption={SQL Query for addresses}]
SELECT DISTINCT ON (a.place_id)
            a.full_json #>> '{5,short_name}' AS state
          , a.full_json #>> '{7,long_name}'  AS postal_code
          , a.full_json #>> '{3,short_name}' AS city
          FROM   addresses_autocomplete a
          JOIN   regions r ON r.short_name = a.full_json #>> '{5,short_name}'
          WHERE  a.formatted = $1
          AND    json_array_length(a.full_json) > 7
          AND    r.country_id = 1
\end{lstlisting}

\subsection{Publishing Dart Packages With Docker}

In order to publish a package in dart you need to authenticate with a web browser. This is easily done if dart is installed locally, however if you are using dart in docker this becomes a bit more confusing.
After authentication with google, copy the response url and curl it within the docker container.

`pub publish'

Go to url in web browser, copy url directed at `localhost` and curl inside docker container.

I believe pub publish spins out a http listener at a localhost port which is why using curl to hit it works in dart.

% Main file dump location

Learned that `::1` means localhost for ip 6, and express has a useful function called req.ip to get client ip, but fails if you are using a proxy.

\begin{lstlisting}[caption={Javascript Program get ip address}]
var ip = req.headers['x-forwarded-for'] || 
     req.connection.remoteAddress || 
     req.socket.remoteAddress ||
     (req.connection.socket ? req.connection.socket.remoteAddress : null);
\end{lstlisting}

\begin{lstlisting}[caption={Javascript Program HTTP server}]
var http = require("http"),
    url = require("url"),
    path = require("path"),
    fs = require("fs"),
    port = process.argv[2] || 8888;

http.createServer(function(request, response) {

  var uri = url.parse(request.url).pathname
    , filename = path.join(process.cwd(), uri);

  fs.exists(filename, function(exists) {
    if(!exists) {
      response.writeHead(404, {"Content-Type": "text/plain"});
      response.write("404 Not Found\n");
      response.end();
      return;
    }

    if (fs.statSync(filename).isDirectory()) filename += '/index.html';

    fs.readFile(filename, "binary", function(err, file) {
      if(err) {        
        response.writeHead(500, {"Content-Type": "text/plain"});
        response.write(err + "\n");
        response.end();
        return;
      }

      response.writeHead(200);
      response.write(file, "binary");
      response.end();
    });
  });
}).listen(parseInt(port, 10));
\end{lstlisting}

\begin{lstlisting}[language=C++, caption={Cpp Testing}]
#include <iostream>
\end{lstlisting}



\paragraph{Thoughts}

Setting on felgo cloud builds was a pain, but somehow I managed to do it.

Need to tag that particular release, the problem was a mix of license code, the way numbering works for felgo among other things.

The issue was plugins I did not need, but were added in.

Looking into admob for google suggests I need around 2k view/downloads to make money,

Will shove it in hopefully for something to happen.

Kinda lazy with the felgo games.



Get https://switch-xci.com/fire-emblem-three-houses-switch-nsp-xci

Download Utorrent

Download Yuzu

https://www.youtube.com/watch?v=bGNBnPSNa8c

Install Virtual Audio Cable

https://www.microsoft.com/en-us/download/details.aspx?id=48145
\section{Code Commands}\label{code-commands}

This is the list of commands that I have careful stored

\subsection{Other Commands}\label{other-commands}

\begin{Shaded}
\NormalTok{vendor/bin/phpunit --coverage-html docs --testsuite FeatureV4}
\end{Shaded}

\begin{verbatim}
sudo -i

root@dockerhost:~# nc -l 443
\end{verbatim}

\begin{verbatim}
echo | <command>
echo | surge
\end{verbatim}

\begin{quote}
Deploys the app to surge without any prompts
\end{quote}

\hypertarget{docker-commands}{%
\subsection{Docker Commands}\label{docker-commands}}

\begin{verbatim}
\end{verbatim}

\hypertarget{bash}{%
\subsection{Bash}\label{bash}}

\begin{quote}
printenv command -- Print all or part of environment
\end{quote}

\begin{verbatim}
sed -e '/pattern="MVG"/s/(^.*$)/<!--\1-->/' policy.xml > policy_test.xml
<policy domain="coder" rights="none" pattern="MVG" />
<!--  <policy domain="coder" rights="none" pattern="MVG" />-->
\end{verbatim}

\begin{verbatim}
sed -i -e 's/rights="none" pattern="PDF"/rights="read|write" pattern="PDF"/g' policy_test.xml
\end{verbatim}

\subsection{Windows}\label{windows}

\begin{verbatim}
taskkill /im node.exe /F
refreshenv
\end{verbatim}

where ping Taskkill is a built-in windows command that, well, kills a
task! the /im flag stands for "imagename", and it lets you pass in the
image name of the process to be terminated which is node.exe in this
case. The /F flag is optional and forces termination of the process.


\subsection{Gitlab}\label{gitlab}

\url{https://gitlab.com/profile/pipeline_quota}

How many minutes I have used, lol heroku-ping doesn't register, maybe it
occurs too quickly, or pages doesn't count as CI.

\subsection{BitBucket}\label{bitbucket}

main account 012345 pass: bchain050

\subsubsection{Git}\label{git}

Undo changes made in a specific folder in Git

\begin{verbatim}
git checkout <folder 
\end{verbatim}

\begin{verbatim}
\end{verbatim}


\subsection{MegaTools}\label{megatools}

All Books I find "useful" will get dumped to my mega account using the
shell tools.

All remote directories start with /Root

\begin{verbatim}
megals -u <USERNAME> -p <PASSWORD>
megacopy -u <USERNAME> -p <PASSWORD> -l <LOCAL_DIR> -r <REMOTE_DIR>
\end{verbatim}

\subsection{GCP Useful Commands}\label{gcp-useful-commands}

\begin{verbatim}
# Creating a new python environment 
# See https://conda.io/docs/user-guide/tasks/manage-python.html#installing-a-different-version-of-python
conda create -n py36 python=3.6 anaconda
source activate py36
#Installing a new package 
conda install -c omnia hmmlearn
# Finding the list of environments for conda 
conda list 
# To install new kernel for juypter notebook 
python -m ipykernel install --user --name myenv --display-name "Python (myenv)"
\end{verbatim}

\subsection{Kubernates}\label{kubernates}

\subsection{Python Commands}\label{python-commands}


\subsection{Gitlab Runner Config}\label{gitlab-runner-config}

Remember to

\begin{verbatim}
sudo chmod +x /etc/gitlab-runner/config.toml
\end{verbatim}


\subsection{Node}\label{node}

\begin{verbatim}
npm cache clean -f
\end{verbatim}


\subsection{Ubuntu \& Linux}\label{ubuntu-linux}

Get csv nth csv column, for first column and quotes.csv

..code-block:: awk -F ""\emph{,"}" '\{print \$1 ","\}' quotes.csv

STorage at
C:UsersstudeAppDataLocalPackagesCanonicalGroupLimited.Ubuntu18.04onWindows\_79rhkp1fndgscLocalStaterootfs

or
C:UsersstudeAppDataLocalPackages\textless DISTRO\textgreater LocalStaterootfs

Yes it is possible, just redirect the output to a file:

..code-block:: shell SomeCommand \textgreater{} SomeFile.txt

Or if you want to append data:

..code-block:: shell SomeCommand \textgreater\textgreater{} SomeFile.txt

..code-block:: shell taskkill /im node.exe

\begin{verbatim}
\end{verbatim}

\begin{verbatim}
\end{verbatim}


\section{Commands}\label{commands}

\begin{Shaded}
\NormalTok{vendor/bin/phpunit --coverage-html docs --testsuite FeatureV4}
\end{Shaded}

\begin{verbatim}
sudo -i

root@dockerhost:~# nc -l 443
\end{verbatim}


\section{Scripts and Tools}\label{scripts-and-tools}


\subsection{Git Tips}\label{git-tips}

I use github for personal repos

gitlab for things that should be private

My bitbucket account is linked to the after graduation gmail account
\href{mailto:davidli012345@gmail.com}{\nolinkurl{davidli012345@gmail.com}}
and bitbucket is the dumping ground for source code I want to examine in
the future.

For git bash the ssh key is at

C:UserswuAppDataRoamingSPB\_Data.ssh

See
\href{https://serverfault.com/questions/50775/how-do-i-change-my-private-key-passphrase}{Private key passphrase}


\subsubsection{Postgres}\label{postgres}

To start running postgres server

\begin{verbatim}
pg_ctl.exe restart -D "C:\Program Files\PostgreSQL\9.6\data"
\end{verbatim}

to quit psql Type q and then press ENTER

Ctrl + D is what I usually use to exit psql console.


\paragraph{Making DB}\label{making-db}

\begin{verbatim}
postgres=# CREATE DATABASE test;
CREATE DATABASE
postgres=# \c test 
\dt 
\q
\end{verbatim}


\subsubsection{Indeed-Job-Scrapper}\label{indeed-job-scrapper}

Three Repos:

\begin{itemize}

\item
  Indeed-Script on Gitlab student gmail account/
\item
  Client on Github Public Folder
\item
  Go Rest API on Github Public Folder (Private for now)
\end{itemize}


\subsection{Uvic Scripts/Tools}\label{uvic-scriptstools}

\textbf{Requirements}

\begin{enumerate}
\def\labelenumi{\arabic{enumi}.}

\item
  This program uses selenium to navigate and extract html from webpages,
  then performs analysis with pandas and nltk.
\item
  If anything breaks just add a time.sleep (5) because when I was
  testing this script, I tested it a lot, and files were stored on the
  cache, resulting a quicker load time.
\end{enumerate}

\begin{longtable}[]{@{}ll@{}}
\toprule
\begin{minipage}[b]{0.20\columnwidth}\raggedright
info\strut
\end{minipage} & \begin{minipage}[b]{0.34\columnwidth}\raggedright
value\strut
\end{minipage}\tabularnewline
\midrule
\endhead
\begin{minipage}[t]{0.20\columnwidth}\raggedright
author:\strut
\end{minipage} & \begin{minipage}[t]{0.34\columnwidth}\raggedright
David Li\strut
\end{minipage}\tabularnewline
\begin{minipage}[t]{0.20\columnwidth}\raggedright
Date:\strut
\end{minipage} & \begin{minipage}[t]{0.34\columnwidth}\raggedright
Jan 07, 2018\strut
\end{minipage}\tabularnewline
\begin{minipage}[t]{0.20\columnwidth}\raggedright
File Name:\strut
\end{minipage} & \begin{minipage}[t]{0.34\columnwidth}\raggedright
lim\_job\_scrapper.py\strut
\end{minipage}\tabularnewline
\bottomrule
\end{longtable}

Summary: Extract metadata from job postings, keywords and save job
postings as static html pages. Requires the user can access
\url{https://learninginmotion.uvic.ca} and submitted their "pink slip"
see
(\url{https://web.uvic.ca/calendar2018-01/undergrad/engineering/co-op.html})
for details.

This program will log into lim using an automated browser, then: 1. find
job postings, 2. extract metadata for job Postings. 3. create a main
html file containing metadata that is linked to individual job postings.

Arguments passed in: 1. UVIC netlink ID 2. UVIC password 3. Co-op term
of interest. For example current( i.e, taking classes in the spring,
looking for summer co-op), future, past

Output: Produces one index.html page containing relative links to
individual job postings, keywords and links to the LIM webpage. Requires
selenium, BeautifulSoup and pandas.

As shown in the screenshot below the value of current depends where you
are at in the term, for example, at May 2018, current respresents the
preceeding term, Fall 2018, but a few weeks ago, we were still at Summer
2018.

Future improvements could include using a {[}STRIKEOUT:jinja template to
improve the appearance of the html pages, would need to add css, also
searchable/sortable tables would be helpful.{]}

See \href{https://web.uvic.ca/~lidavid/jobScrapping/LIMScrap/}{Uvic Job
Postings (Jan 01, 2018)} for sample output.


\subsubsection{Other Considerations}\label{other-considerations}

\paragraph{Future improvements}\label{future-improvements}

\begin{itemize}

\item
  {[}STRIKEOUT:Adding a summary of results (could be a plot).{]}
\item
  Determining if a job is suitable based on keywords ( text analysis).
\item
  {[}STRIKEOUT:Implementing searching/sortable tables.{]}
\item
  {[}STRIKEOUT:Improving the appearance of HTML tables by rendering a
  template (Jinja){]}
\end{itemize}


\subparagraph{Nice to Have}\label{nice-to-have}

\begin{itemize}

\item
  Summarizating the job posting.
\item
  Print out all html files as pdfs.
\end{itemize}


\subsubsection{Running the Script}\label{running-the-script}

After navigating to the directory of the script and installing the
necessary packages (if needed). The following commands can be used to
run the script.

Also, it will count the number of jobs vs the date

\begin{longtable}[]{@{}ll@{}}
\toprule
\begin{minipage}[b]{0.20\columnwidth}\raggedright
Date\strut
\end{minipage} & \begin{minipage}[b]{0.21\columnwidth}\raggedright
Num of jobs\strut
\end{minipage}\tabularnewline
\midrule
\endhead
\begin{minipage}[t]{0.20\columnwidth}\raggedright
2018-04-26\strut
\end{minipage} & \begin{minipage}[t]{0.21\columnwidth}\raggedright
39\strut
\end{minipage}\tabularnewline
\bottomrule
\end{longtable}

\begin{Shaded}
python  lim\_job\_scrapper.py netlinkid netlinkPass term year 
\end{Shaded}

Where the arguments are: * netlinkid and netlinkPass are uvic login
information * term is either {[}Current, Future, Past{]}

\subsection{Adding apps to bitbucket}\label{adding-apps-to-bitbucket}

\begin{verbatim}
git add remote bitbucket git@bitbucket.org:grandfleet/<name of repo>
git push bitbucket master
\end{verbatim}


\subsection{ENGR 003} 

Decentralized Application Developer

\begin{enumerate}
\def\labelenumi{\arabic{enumi}.}
\item
  \begin{description}
  \item[Tools:]
  \begin{itemize}
  
  \item
    Google Drive
  \item
    Trello
  \item
    Slack
  \item
    Gitlab
  \item
    Microsoft Word
  \end{itemize}
  \end{description}
\item
  \begin{description}
  \item[JavaScript]
  \begin{itemize}
  
  \item
    Working with webpack, node, package management with npm and yarn
  \item
    Creation of React components using props and context, prop-types,
    etc ...
  \item
    React Router, arrow functions, promise chains
  \item
    HTML + CSS + JSX + REDUX
  \item
    JSDOC documentation
  \end{itemize}
  \end{description}
\item
  \begin{description}
  \item[Solidity]
  \begin{itemize}
  
  \item
    Knowledge of ERC20 and ERC721 token standards
  \item
    Revert, Assert, Requires statements and how to use them
  \item
    Optimization of smart contracts
  \item
    Deploying to test nets
  \item
    Setting up CI for truffle using gitlab, circleCI and travisCI
  \end{itemize}
  \end{description}
\item
  \begin{description}
  \item[Databases]
  \begin{itemize}
  
  \item
    postgres (designing database schemas), advanced open source database
  \item
    javascript libraries, pg-promise and automated tests
  \item
    RESTFUL API
  \end{itemize}
  \end{description}
\item
  \begin{description}
  \item[CI/CD]
  \begin{itemize}
  
  \item
    SSH and configuration of remote servers
  \item
    Docker, working with docker and dockerizing application
  \item
    Script Scripting in Bash
  \item
    Automated testing with jest, API testing with superagent
  \end{itemize}
  \end{description}
\item
  \begin{description}
  \item[Windows]
  \begin{itemize}
  
  \item
    Choco, Scoop
  \item
    Boostnote and/or latex
  \end{itemize}
  \end{description}
\end{enumerate}


\subsection{ENGR 001 \& ENGR 002}\label{engr-001-engr-002}

Data Analyst

The tools I used in this co-op include:

\begin{enumerate}
\def\labelenumi{\arabic{enumi}.}
\item
  \begin{description}
  \item[Atlassian Tools:]
  \begin{enumerate}
  \def\labelenumii{\arabic{enumii}.}
  
  \item
    Confluence -\/-\/- wiki
  \item
    JIRA -\/-\/- issue tracking tool
  \end{enumerate}
  \end{description}
\item
  \begin{description}
  \item[Microsoft Tools:]
  \begin{enumerate}
  \def\labelenumii{\arabic{enumii}.}
  
  \item
    Microsoft Word
  \item
    Microsoft Excel
  \item
    Microsoft PowerPoint
  \item
    Microsoft SharePoint
  \item
    Microsoft Visio
  \item
    Microsoft Outlook
  \item
    Microsoft Lync
  \end{enumerate}
  \end{description}
\item
  Oracle Databases
\item
  Java \& Powershell
\item
  VbScript
\end{enumerate}


\subsection{2018 List}\label{list}


\subsubsection{Version Control \& CI/CD \&
Shell}\label{version-control-cicd-shell}

\begin{itemize}

\item
  Github, Bitbucket, Gitlab
\item
  Travis, CircleCI, Gitlab
\item
  Docker, Bash, OSX
\end{itemize}


\subsubsection{Productivity Tools}\label{productivity-tools}

\begin{itemize}

\item
  Slack, Drive
\item
  Discord, !Rocketchat
\end{itemize}


\subsubsection{Documentation}\label{documentation}

\begin{itemize}

\item
  Mkdocs, Sphinx, Jsdoc
\item
  Boostnote/Vuepress, Hugo, lwarp/latex
\item
  Wikijs, frozen-flask, docusaurus
\item
  Explore !git notebook
  (\url{https://thomasreinecke.github.io/git-playbook/\#/playbook})
\end{itemize}


\subsubsection{Blockchain}\label{blockchain}

\begin{itemize}

\item
  Ethereum, Truffle, Ganache
\item
  Drizzle, Solidity, Solidity-docgen
\item
  Infura, high stackvoerflow account
\item
  !Hashgraph
\end{itemize}


\subsubsection{Programming Languages}\label{programming-languages}

\begin{itemize}

\item
  Javascript, Python, Latex
\item
  Matlab, C, C\#, Java
\item
  PHP, Perl, Node, Go
\end{itemize}

Markup Stuff

\begin{itemize}

\item
  HTML, CSS
\item
  Markdown, RST
\end{itemize}


\subsubsection{Databases}\label{databases}

Mongo, Postgres, MySQL, Oracle


\subsubsection{CMD Tools}\label{cmd-tools}

Yarn, npm, pandoc, etc .. imagemagick

\hypertarget{hugo-tools-to-consider}{%
\subsection{Hugo Tools to Consider}\label{hugo-tools-to-consider}}

\begin{enumerate}
\def\labelenumi{\arabic{enumi}.}

\item
  \url{https://themes.gohugo.io/theme/hugo-assembly/\#contact}
\item
  \url{https://github.com/avelino/hugo-theme-sarah}
\item
  \url{https://www.valentinog.com/blog/webpack-tutorial/}
\item
  \url{https://medium.freecodecamp.org/why-react16-is-a-blessing-to-react-developers-31433bfc210a}
\end{enumerate}
\begin{mdframed}[style=theorem, frametitle={Algorithm Features}]
	A good algorithm must have three features: correctness, maintainability, and
effciency.
\end{mdframed}

\begin{mdframed}[style=important, frametitle={Data Structures}]
The term data structure refers to the organization of data in a
computer's memory, in order to retrieve it quickly for processing.
It is a scheme for data organization to decouple the functional
definition of a data structure from its implementation. A data
structure is chosen based on the problem type and the
operations performed on the data.
\end{mdframed}

There are five basic rules for calculating an algorithm’s Big O notation.

\begin{enumerate}
\item If an algorithm performs a certain sequence of steps f(N) times for a mathematical function f, then it takes O(f(N)) steps.
\item  If an algorithm performs an operation that takes O(f(N)) steps and then
performs a second operation that takes O(g(N)) steps for functions f and
g, then the algorithm’s total performance is O(f(N) g(N)).
\item  If an algorithm takes $O(f(N)+g(N))$ time and the function f(N) is greater
than g(N) for large N, then the algorithm’s performance can be simplified
to $O(f(N))$.
\item  If an algorithm performs an operation that takes $O(f(N))$ steps, and for
every step in that operation it performs another $O(g(N))$ steps, then the
algorithm’s total performance is $O(f(N)\times g(N)$.
\item  Ignore constant multiples. If C is a constant, $O(C \times f(N))$ is the same as
$O(f(N))$, and $O(C \times f(N))$ is the same as O(f(N)).

Adapter, bridge, composite,
decorator, facade, flyweight, private class data, and proxy are the
Gang of Four (GoF) structural design patterns. 

The adapter pattern comprises the target, adaptee, adapter, and
client:
\begin{itemize}
\item Target is the interface that the client calls and invokes
methods on the adapter and adaptee.
\item The client wants the incompatible interface implemented
by the adapter.
\item The adapter translates the incompatible interface of the
adaptee into an interface that the client wants.
\end{itemize}

\lstinputlisting[caption={Golang Adaptor Example}]{chapters/important/algorithms/adaptor.go}

\textbf{Bridge}
Bridge decouples the implementation from the abstraction. The
abstract base class can be subclassed to provide different
implementations and allow implementation details to be
modified easily. The interface, which is a bridge, helps in making
the functionality of concrete classes independent from the
interface implementer classes. The bridge patterns allow the
implementation details to change at runtime.

\lstinputlisting[caption={Golang Adaptor Example}]{chapters/important/algorithms/bridge.go}


\begin{mdframed}[style=important, frametitle={Composite}]
A composite is a group of similar objects in a single object.
Objects are stored in a tree form to persist the whole hierarchy.
The composite pattern is used to change a hierarchical collection
of objects. 
\end{mdframed}

The composite pattern comprises the component interface, component
class, composite, and client:
\begin{itemize}
	\item The component interface defines the default behavior of all
	objects and behaviors for accessing the components of
	the composite.
	\item  The composite and component classes implement the component
	interface.
	\item The client interacts with the component interface to
	invoke methods in the composite.
\end{itemize}

\begin{mdframed}[style=important, frametitle={Decorator}]
In a scenario where class responsibilities are removed or added,
the decorator pattern is applied. The decorator pattern helps
with subclassing when modifying functionality, instead of static
inheritance. An object can have multiple decorators and runtime decorators. The single responsibility principle can be
achieved using a decorator.
\end{mdframed}

% Note to self, refer to all go programs are go programs, golang is reserved for the code.
\begin{lstlisting}[caption={Golang Program}]
package main

import (
	"fmt"
	"io/ioutil"
)
\end{lstlisting}


\begin{warpprint}   % For print output ...
\cleardoublepage    % ... a common method to place index entry into TOC.
\phantomsection
\addcontentsline{toc}{chapter}{\indexname}
\end{warpprint}
\ForceHTMLPage      % HTML index will be on its own page.
\ForceHTMLTOC       % HTML index will have its own toc entry.
\printindex

\nocite{*}
\printbibliography

\glsaddall
\printglossaries

\end{document}