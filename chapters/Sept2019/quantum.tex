

\begin{mdframed}[style=important, frametitle={QPU Versus GPU: Some Common Characteristics}]

\begin{itemize}
\item It is very rare that a program will run entirely on a QPU. Usually, a program running on a CPU will issue QPU instructions, and later retrieve the results.
\item Some tasks are very well suited to the QPU, and others are not.
\item The QPU runs on a separate clock from the CPU, and usually has its own dedicated hardware interfaces to external devices (such as optical outputs).
\item A typical QPU has its own special RAM, which the CPU cannot efficiently access.
\item Here's a list of pertinent facts about what it’s like to program a QPU:
It is very rare that a program will run entirely on a QPU. Usually, a program running on a CPU will issue QPU instructions, and later retrieve the results.
Some tasks are very well suited to the QPU, and others are not.
The QPU runs on a separate clock from the CPU, and usually has its own dedicated hardware interfaces to external devices (such as optical outputs).
A typical QPU has its own special RAM, which the CPU cannot efficiently access.
\end{itemize}
\end{mdframed}

% Having java here, doesn't matter because the language gets mapped in the html anyway using prism
\begin{lstlisting}[language=java, caption=Javascript Program: Quantum random spy hunter]
Example 2-4. Quantum random spy hunter
qc.reset(3);
qc.discard();
var a = qint.new(1, 'alice');
var fiber = qint.new(1, 'fiber');
var b = qint.new(1, 'bob');

function random_bit(q) {
    q.write(0);
    q.had();
    return q.read();
}

// Generate two random bits
var send_had = random_bit(a);
var send_val = random_bit(a);

// Prepare Alice's qubit
a.write(0);
if (send_val)  // Use a random bit to set the value
    a.not();
if (send_had)  // Use a random bit to apply HAD or not
    a.had();

// Send the qubit!
fiber.exchange(a);

// Activate the spy
var spy_is_present = true;
\end{lstlisting}