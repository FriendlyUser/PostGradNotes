

\subsubsection{Indeed-Job-Scrapper}\label{indeed-job-scrapper}

Three Repos:

\begin{itemize}

\item
  Indeed-Script on Gitlab student gmail account/
\item
  Client on Github Public Folder
\item
  Go Rest API on Github Public Folder (Private for now)
\end{itemize}


\subsection{Uvic Scripts/Tools}\label{uvic-scriptstools}

\textbf{Requirements}

\begin{enumerate}
\def\labelenumi{\arabic{enumi}.}

\item
  This program uses selenium to navigate and extract html from webpages,
  then performs analysis with pandas and nltk.
\item
  If anything breaks just add a time.sleep (5) because when I was
  testing this script, I tested it a lot, and files were stored on the
  cache, resulting a quicker load time.
\end{enumerate}

\begin{longtable}[]{@{}ll@{}}
\toprule
\begin{minipage}[b]{0.20\columnwidth}\raggedright
info\strut
\end{minipage} & \begin{minipage}[b]{0.34\columnwidth}\raggedright
value\strut
\end{minipage}\tabularnewline
\midrule
\endhead
\begin{minipage}[t]{0.20\columnwidth}\raggedright
author:\strut
\end{minipage} & \begin{minipage}[t]{0.34\columnwidth}\raggedright
David Li\strut
\end{minipage}\tabularnewline
\begin{minipage}[t]{0.20\columnwidth}\raggedright
Date:\strut
\end{minipage} & \begin{minipage}[t]{0.34\columnwidth}\raggedright
Jan 07, 2018\strut
\end{minipage}\tabularnewline
\begin{minipage}[t]{0.20\columnwidth}\raggedright
File Name:\strut
\end{minipage} & \begin{minipage}[t]{0.34\columnwidth}\raggedright
lim\_job\_scrapper.py\strut
\end{minipage}\tabularnewline
\bottomrule
\end{longtable}

Summary: Extract metadata from job postings, keywords and save job
postings as static html pages. Requires the user can access
\url{https://learninginmotion.uvic.ca} and submitted their "pink slip"
see
(\url{https://web.uvic.ca/calendar2018-01/undergrad/engineering/co-op.html})
for details.

This program will log into lim using an automated browser, then: 1. find
job postings, 2. extract metadata for job Postings. 3. create a main
html file containing metadata that is linked to individual job postings.

Arguments passed in: 1. UVIC netlink ID 2. UVIC password 3. Co-op term
of interest. For example current( i.e, taking classes in the spring,
looking for summer co-op), future, past

Output: Produces one index.html page containing relative links to
individual job postings, keywords and links to the LIM webpage. Requires
selenium, BeautifulSoup and pandas.

As shown in the screenshot below the value of current depends where you
are at in the term, for example, at May 2018, current respresents the
preceeding term, Fall 2018, but a few weeks ago, we were still at Summer
2018.

Future improvements could include using a {[}STRIKEOUT:jinja template to
improve the appearance of the html pages, would need to add css, also
searchable/sortable tables would be helpful.{]}

See \href{https://web.uvic.ca/~lidavid/jobScrapping/LIMScrap/}{Uvic Job
Postings (Jan 01, 2018)} for sample output.


\subsubsection{Other Considerations}\label{other-considerations}

\paragraph{Future improvements}\label{future-improvements}

\begin{itemize}

\item
  {[}STRIKEOUT:Adding a summary of results (could be a plot).{]}
\item
  Determining if a job is suitable based on keywords ( text analysis).
\item
  {[}STRIKEOUT:Implementing searching/sortable tables.{]}
\item
  {[}STRIKEOUT:Improving the appearance of HTML tables by rendering a
  template (Jinja){]}
\end{itemize}


\subparagraph{Nice to Have}\label{nice-to-have}

\begin{itemize}

\item
  Summarizating the job posting.
\item
  Print out all html files as pdfs.
\end{itemize}


\subsubsection{Running the Script}\label{running-the-script}

After navigating to the directory of the script and installing the
necessary packages (if needed). The following commands can be used to
run the script.

Also, it will count the number of jobs vs the date

\begin{longtable}[]{@{}ll@{}}
\toprule
\begin{minipage}[b]{0.20\columnwidth}\raggedright
Date\strut
\end{minipage} & \begin{minipage}[b]{0.21\columnwidth}\raggedright
Num of jobs\strut
\end{minipage}\tabularnewline
\midrule
\endhead
\begin{minipage}[t]{0.20\columnwidth}\raggedright
2018-04-26\strut
\end{minipage} & \begin{minipage}[t]{0.21\columnwidth}\raggedright
39\strut
\end{minipage}\tabularnewline
\bottomrule
\end{longtable}

\begin{Shaded}
python  lim\_job\_scrapper.py netlinkid netlinkPass term year 
\end{Shaded}

Where the arguments are: * netlinkid and netlinkPass are uvic login
information * term is either {[}Current, Future, Past{]}

\subsection{Adding apps to bitbucket}\label{adding-apps-to-bitbucket}

\begin{verbatim}
git add remote bitbucket git@bitbucket.org:grandfleet/<name of repo>
git push bitbucket master
\end{verbatim}


\subsection{ENGR 003} 

Decentralized Application Developer

\begin{enumerate}
\def\labelenumi{\arabic{enumi}.}
\item
  \begin{description}
  \item[Tools:]
  \begin{itemize}
  
  \item
    Google Drive
  \item
    Trello
  \item
    Slack
  \item
    Gitlab
  \item
    Microsoft Word
  \end{itemize}
  \end{description}
\item
  \begin{description}
  \item[JavaScript]
  \begin{itemize}
  
  \item
    Working with webpack, node, package management with npm and yarn
  \item
    Creation of React components using props and context, prop-types,
    etc ...
  \item
    React Router, arrow functions, promise chains
  \item
    HTML + CSS + JSX + REDUX
  \item
    JSDOC documentation
  \end{itemize}
  \end{description}
\item
  \begin{description}
  \item[Solidity]
  \begin{itemize}
  
  \item
    Knowledge of ERC20 and ERC721 token standards
  \item
    Revert, Assert, Requires statements and how to use them
  \item
    Optimization of smart contracts
  \item
    Deploying to test nets
  \item
    Setting up CI for truffle using gitlab, circleCI and travisCI
  \end{itemize}
  \end{description}
\item
  \begin{description}
  \item[Databases]
  \begin{itemize}
  
  \item
    postgres (designing database schemas), advanced open source database
  \item
    javascript libraries, pg-promise and automated tests
  \item
    RESTFUL API
  \end{itemize}
  \end{description}
\item
  \begin{description}
  \item[CI/CD]
  \begin{itemize}
  
  \item
    SSH and configuration of remote servers
  \item
    Docker, working with docker and dockerizing application
  \item
    Script Scripting in Bash
  \item
    Automated testing with jest, API testing with superagent
  \end{itemize}
  \end{description}
\item
  \begin{description}
  \item[Windows]
  \begin{itemize}
  
  \item
    Choco, Scoop
  \item
    Boostnote and/or latex
  \end{itemize}
  \end{description}
\end{enumerate}


\subsection{ENGR 001 \& ENGR 002}\label{engr-001-engr-002}

Data Analyst

The tools I used in this co-op include:

\begin{enumerate}
\def\labelenumi{\arabic{enumi}.}
\item
  \begin{description}
  \item[Atlassian Tools:]
  \begin{enumerate}
  \def\labelenumii{\arabic{enumii}.}
  
  \item
    Confluence -\/-\/- wiki
  \item
    JIRA -\/-\/- issue tracking tool
  \end{enumerate}
  \end{description}
\item
  \begin{description}
  \item[Microsoft Tools:]
  \begin{enumerate}
  \def\labelenumii{\arabic{enumii}.}
  
  \item
    Microsoft Word
  \item
    Microsoft Excel
  \item
    Microsoft PowerPoint
  \item
    Microsoft SharePoint
  \item
    Microsoft Visio
  \item
    Microsoft Outlook
  \item
    Microsoft Lync
  \end{enumerate}
  \end{description}
\item
  Oracle Databases
\item
  Java \& Powershell
\item
  VbScript
\end{enumerate}


\subsection{2018 List}\label{list}


\subsubsection{Version Control \& CI/CD \&
Shell}\label{version-control-cicd-shell}

\begin{itemize}

\item
  Github, Bitbucket, Gitlab
\item
  Travis, CircleCI, Gitlab
\item
  Docker, Bash, OSX
\end{itemize}


\subsubsection{Productivity Tools}\label{productivity-tools}

\begin{itemize}

\item
  Slack, Drive
\item
  Discord, !Rocketchat
\end{itemize}


\subsubsection{Documentation}\label{documentation}

\begin{itemize}

\item
  Mkdocs, Sphinx, Jsdoc
\item
  Boostnote/Vuepress, Hugo, lwarp/latex
\item
  Wikijs, frozen-flask, docusaurus
\item
  Explore !git notebook
  (\url{https://thomasreinecke.github.io/git-playbook/\#/playbook})
\end{itemize}


\subsubsection{Blockchain}\label{blockchain}

\begin{itemize}

\item
  Ethereum, Truffle, Ganache
\item
  Drizzle, Solidity, Solidity-docgen
\item
  Infura, high stackvoerflow account
\item
  !Hashgraph
\end{itemize}


\subsubsection{Programming Languages}\label{programming-languages}

\begin{itemize}

\item
  Javascript, Python, Latex
\item
  Matlab, C, C\#, Java
\item
  PHP, Perl, Node, Go
\end{itemize}

Markup Stuff

\begin{itemize}

\item
  HTML, CSS
\item
  Markdown, RST
\end{itemize}


\subsubsection{Databases}\label{databases}

Mongo, Postgres, MySQL, Oracle


\subsubsection{CMD Tools}\label{cmd-tools}

Yarn, npm, pandoc, etc .. imagemagick

\hypertarget{hugo-tools-to-consider}{%
\subsection{Hugo Tools to Consider}\label{hugo-tools-to-consider}}

\begin{enumerate}
\def\labelenumi{\arabic{enumi}.}

\item
  \url{https://themes.gohugo.io/theme/hugo-assembly/\#contact}
\item
  \url{https://github.com/avelino/hugo-theme-sarah}
\item
  \url{https://www.valentinog.com/blog/webpack-tutorial/}
\item
  \url{https://medium.freecodecamp.org/why-react16-is-a-blessing-to-react-developers-31433bfc210a}
\end{enumerate}