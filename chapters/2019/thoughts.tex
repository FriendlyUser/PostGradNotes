
\textbf{Magneta Music Generation} Some of the beats generated were acceptable, could make sound effects from randomized shorter effects.

Probably just trial and error as for images, swipe them off the internet, because I still need animations for now.


\section{SQL}

Query
\begin{lstlisting}[caption={SQL Query for length}]
SELECT first_name, length(first_name) as first_length, last_name, length(last_name) as last_length, length(first_name)+length(last_name) as total_length FROM users WHERE length(first_name)+length(last_name) > 30;
\end{lstlisting}

\begin{lstlisting}[caption={SQL Query for addresses}]
SELECT DISTINCT ON (a.place_id)
            a.full_json #>> '{5,short_name}' AS state
          , a.full_json #>> '{7,long_name}'  AS postal_code
          , a.full_json #>> '{3,short_name}' AS city
          FROM   addresses_autocomplete a
          JOIN   regions r ON r.short_name = a.full_json #>> '{5,short_name}'
          WHERE  a.formatted = $1
          AND    json_array_length(a.full_json) > 7
          AND    r.country_id = 1
\end{lstlisting}

\subsection{Publishing Dart Packages With Docker}

In order to publish a package in dart you need to authenticate with a web browser. This is easily done if dart is installed locally, however if you are using dart in docker this becomes a bit more confusing.
After authentication with google, copy the response url and curl it within the docker container.

`pub publish'

Go to url in web browser, copy url directed at `localhost` and curl inside docker container.

I believe pub publish spins out a http listener at a localhost port which is why using curl to hit it works in dart.

% Main file dump location

Learned that `::1` means localhost for ip 6, and express has a useful function called req.ip to get client ip, but fails if you are using a proxy.

\begin{lstlisting}[caption={Javascript Program get ip address}]
var ip = req.headers['x-forwarded-for'] || 
     req.connection.remoteAddress || 
     req.socket.remoteAddress ||
     (req.connection.socket ? req.connection.socket.remoteAddress : null);
\end{lstlisting}

\begin{lstlisting}[caption={Javascript Program HTTP server}]
var http = require("http"),
    url = require("url"),
    path = require("path"),
    fs = require("fs"),
    port = process.argv[2] || 8888;

http.createServer(function(request, response) {

  var uri = url.parse(request.url).pathname
    , filename = path.join(process.cwd(), uri);

  fs.exists(filename, function(exists) {
    if(!exists) {
      response.writeHead(404, {"Content-Type": "text/plain"});
      response.write("404 Not Found\n");
      response.end();
      return;
    }

    if (fs.statSync(filename).isDirectory()) filename += '/index.html';

    fs.readFile(filename, "binary", function(err, file) {
      if(err) {        
        response.writeHead(500, {"Content-Type": "text/plain"});
        response.write(err + "\n");
        response.end();
        return;
      }

      response.writeHead(200);
      response.write(file, "binary");
      response.end();
    });
  });
}).listen(parseInt(port, 10));
\end{lstlisting}