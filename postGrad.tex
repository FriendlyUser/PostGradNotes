
% Save this as tutorial.tex for the lwarp package tutorial.

% Listings works with book
\documentclass{book}

\usepackage{iftex}

% --- LOAD FONT SELECTION AND ENCODING BEFORE LOADING LWARP ---

\ifPDFTeX
\usepackage{lmodern}            % pdflatex or dvi latex
\usepackage[T1]{fontenc}
\usepackage[utf8]{inputenc}
\else
\usepackage{fontspec}           % XeLaTeX or LuaLaTeX
\fi

% --- LWARP IS LOADED NEXT ---
\usepackage[
HomeHTMLFilename=index,     % Filename of the homepage.
%   HTMLFilename={node-},       % Filename prefix of other pages.
%   IndexLanguage=english,      % Language for xindy index, glossary.
%   latexmk,                    % Use latexmk to compile.
%   OSWindows,                  % Force Windows. (Usually automatic.)
% Mathjax is the way to go, if I lose mathjax, host it myself????? Don't have too much math involved in my equations.
mathjax,                    % Use MathJax to display math.
]{lwarp}
% \boolfalse{FileSectionNames}  % If false, numbers the files.


% --- Custom non tutorial content
% --- Listings
\usepackage{listings}

% Styling only for pdf --- for now
\lstset{language=C++,
	basicstyle=\ttfamily,
	keywordstyle=\color{blue}\ttfamily,
	stringstyle=\color{red}\ttfamily,
	commentstyle=\color{green}\ttfamily,
	morecomment=[l][\color{magenta}]{\#},
	captionpos=t,
}

% --- Glossary Terms
\usepackage[acronym, toc, xindy]{glossaries}
% Hands on Systems Programming C++
\newglossaryentry{api}
{
   name={API},
   description={An Application Programming Interface (API) is a particular set
           of rules and specifications that a software program can follow to access and make use of the services and resources provided by another particular software program that implements that API},
   first={Application Programming Interface (API)},
   long={Application Programming Interface}
}

\newglossaryentry{staticlib}
{
	name={Static libraries},
	description ={
			libraries that are linked at compile time
	}
}

\newglossaryentry{dynamiclib}
{
	name={Dynamic libraries},
	description ={
			libraries that are linked at run time
	}
}

%%% Laravel Notes

\newglossaryentry{depInj}
{
	name={Dependency Injection},
	description={
		Dependency injection means that, rather than being instantiated ("newed up") within a class, each class's dependencies will be injected in from the outside.
	}
}

\newglossaryentry{IoC}
{
	name={IoC},
	description={
		In software engineering, inversion of control (IoC) is a programming principle. IoC inverts the flow control as compared to traditional control flow. In IoC, custom-written portions of a computer program receive the flow of control from a generic framework. A software architecture with this design inverts control as compared to traditional procedural programming: in traditional programming, the custom code that expresses the purpose of the program calls into reusable libraries to take care of generic tasks, but with inversion of control, it is the framework that calls into the custom, or task-specific, code.
	},
	first={Inversion of Control},
	long={Inversion of Control}
}

%% --- GAN Notes ---
\newglossaryentry{GAN}
{
	name={GAN},
	description={
		A GAN is a deep neural network architecture made up of two networks, a generator network and a discriminator network. Through multiple cycles of generation and discrimination, both networks train each other, while simultaneously trying to outwit each other.
	},
	first={Generative Adversarial Networks},
	long={Generative Adversarial Networks}
}
% --- Software Tools ---
\newglossaryentry{git}
{
	name={git},
	description ={
			the most popular version control system used for tracking changes in source code and coordinating work on those files among multiple people
	}
}

% --- Quantum Computing ---


% --- Algorithms ---
\newglossaryentry{Adapter}
{
	name = {Adapter},
	description = {
		The adapter pattern provides a wrapper with an interface
		required by the \gls{api} client to link incompatible types and act as a
		translator between the two types. The adapter uses the interface
		of a class to be a class with another compatible interface. When
		requirements change, there are scenarios where class
		functionality needs to be changed because of incompatible
		interfaces.
	}
}

\newglossaryentry{Bridge}
{
	name={Bridge},
	description={
			Bridge decouples the implementation from the abstraction. The
		abstract base class can be subclassed to provide different
		implementations and allow implementation details to be
		modified easily. 
	}
}

 % Glossary File
\makeglossaries

% Hands on Systems Programming C++
\newacronym{api}{API}{application programming interface}

\newacronym{abi}{ABI}{application binary interface}

\newacronym{qpu}{QPU}{Quantum Processing Unit}
\glsaddall


% --- MDFFRAMED CUSTOMIZATION
\usepackage{mdframed}

\mdfdefinestyle{example}{
	linecolor=blue!50,
	align=center, 
	backgroundcolor=blue!30, 
	frametitlebackgroundcolor=blue!15,
	shadow=true
}
\mdfdefinestyle{definition}{
	linecolor=red!50,
	align=center, 
	backgroundcolor=red!30, 
	frametitlebackgroundcolor=red!15,
	shadow=true
}
\mdfdefinestyle{theorem}{
	linecolor=yellow!50,
	align=center, 
	backgroundcolor=yellow!30, 
	frametitlebackgroundcolor=yellow!15,
	shadow=true
}

\mdfdefinestyle{important}{
	linecolor=purple!50,
	align=center, 
	backgroundcolor=purple!30, 
	frametitlebackgroundcolor=purple!15,
	shadow=true
}

% --- References ---
\newcommand{\onlineCite}{[Online] Available: }	% Used in BIBLIOGRAPHY 
\usepackage[backend=bibtex,sorting=none]{biblatex}	% Sort by citation order, IEEE
\addbibresource{config/references.bib}

% --- MATH LIBRARIES ---
\usepackage{amssymb}
% --- LOAD PDFLATEX MATH FONTS HERE ---

% --- OTHER PACKAGES ARE LOADED AFTER LWARP ---
\usepackage{makeidx} \makeindex
\usepackage{xcolor}             % (Demonstration purposes only.)
\usepackage{hyperref,cleveref}  % LOAD THESE LAST!

% --- LATEX AND HTML CUSTOMIZATION ---
\title{The Lwarp Tutorial}
\author{Some Author}
\setcounter{tocdepth}{2}        % Include subsections in the \TOC.
\setcounter{secnumdepth}{2}     % Number down to subsections.
\setcounter{FileDepth}{1}       % Split \HTML\ files at sections
\booltrue{CombineHigherDepths}  % Combine parts/chapters/sections
\setcounter{SideTOCDepth}{1}    % Include subsections in the side\TOC
\HTMLTitle{Webpage Title}       % Overrides \title for the web page.
\HTMLAuthor{Some Author}        % Sets the HTML meta author tag.
\HTMLLanguage{en-US}            % Sets the HTML meta language.
\HTMLDescription{A description.}% Sets the HTML meta description.
\HTMLFirstPageTop{Name and \fbox{HOMEPAGE LOGO}}
\HTMLPageTop{\fbox{LOGO}}
\HTMLPageBottom{Contact Information and Copyright}
\CSSFilename{lwarp_sagebrush.css}


%%%%%%%%%%%%%%%%%%%%%%%%%%%%%%%%%%%%%%%%%%%%%%%%%%%%%%%%%%%%%%%%%%%%%%%%%%%%%%%%
%%%%%%%%%%%%%%%%%%%%%%%% UTILITY COMMANDS %%%%%%%%%%%%%%%%%%%%%%%%%%%%%%%%%%%%%%
%%%%%%%%%%%%%%%%%%%%%%%%%%%%%%%%%%%%%%%%%%%%%%%%%%%%%%%%%%%%%%%%%%%%%%%%%%%%%%%%

%%%%%%%%%%%%%%%%%%%%%%%%%%%%%%%%%%%%%%%%%%%%%%%%%%%%%%%%%%%%%%%%%%%%%%%%%%%%%%%%%
%%%%%%%%%%%%%%%%%%%%%%%%%%%%%% REFERENCE COMMANDS OF INTEREST %%%%%%%%%%%%%%%%%%%
%%%%%%%%%%%%%%%%%%%%%%%%%%%%%%%%%%%%%%%%%%%%%%%%%%%%%%%%%%%%%%%%%%%%%%%%%%%%%%%%%

%  replacementTerms = ["Cpp Code", cplusplus, "Latex Code", latex, "Python Script", python, 
% "Bash Script", bash,"Matlab Script", matlab,"Yaml File",yaml,"JSON Output", json, "Golang", golang]
% These are the terms that are manually converted to code listings, by default it is converted to cpp, thanks to SENG475

% Other Notes

% --- Mdframed examples
% 	Values include:	example, definition, theorem and important
%
%	\begin{mdframed}[style=definition, frametitle={SENG475 Example Code Listing}]
%		All for the sake of glory city. We see that in \cref{ex:1} that mdframed makes 
%		environments look nice
%	\end{mdframed}
%	
%	\begin{mdframed}[style=theorem, frametitle={SENG475 Example Code Listing}]
%		All for the sake of glory city. We see that in \cref{ex:1} that mdframed makes 
%		environments look nice
%	\end{mdframed}
%	
%	\begin{mdframed}[style=important, frametitle={SENG475 Example Code Listing}]
%		All for the sake of glory city. We see that in \cref{ex:1} that mdframed makes 
%		environments look nice
%	\end{mdframed}
\begin{document}


\begin{warpHTML}
	\maketitle                   % Or titlepage/titlingpage environment.
\end{warpHTML}
\begin{center}
	%		{\vspace*{3pt} }
	{\Huge \textsc{Personal Study Notes} \\ \vspace{40pt}}
	%	{\Huge  \\ \vspace{4pt}}  
	\rule[13pt]{1\textwidth}{1pt} \\ \vspace{1pt}
	{\LARGE \textbf{{\textsc{Software Engineer}}} \\ }
	{\Large \textsc{Post Grad Notes} \\} 
	\vspace{4pt} 
	%{\Large \textsc{Spring 2018}} \\ 
	\vspace{20pt}
	{\Large \textsc{Victoria, British Columbia, Canada} \\ \vspace{45pt} }
	
	%		\\ \vspace{10pt}
	{\Large \textsc{\today} \\ \vspace{15pt}
		\textbf{Name:} \hfill David Li \\
		\textbf{Student Number:} \hfill V00818631 \\
		\textbf{Email:} \hfill \href{mailto:davidli012345@gmail.ca}{davidli012345@gmail.ca} \\
		\vspace{15pt}
		{\Large \textsc{In partial fulfillment of the never ending quest to learn. \\
			}
		}	
	}
\end{center}

% An article abstract would go here.

\tableofcontents                % MUST BE BEFORE THE FIRST SECTION BREAK!
\listoffigures
\lstlistoflistings % THIS SEEMS TO WORK WITH THE BOOK CLASS

\chapter{Example chapter}

\section{A section}

This is some text which is indexed.\index{Some text.}

\subsection{A subsection}

See \cref{fig:withtext}.

\begin{mdframed}[style=definition, frametitle={SENG475 Example Code Listing}]
	All for the sake of glory city. We see that in \cref{ex:1} that mdframed makes 
	environments look nice
\end{mdframed}

\begin{mdframed}[style=theorem, frametitle={SENG475 Example Code Listing}]
	All for the sake of glory city. We see that in \cref{ex:1} that mdframed makes 
	environments look nice
\end{mdframed}

\begin{mdframed}[style=important, frametitle={SENG475 Example Code Listing}]
	All for the sake of glory city. We see that in \cref{ex:1} that mdframed makes 
	environments look nice
\end{mdframed}
	
\begin{figure}\begin{center}
\fbox{\textcolor{blue!50!green}{Text in a figure.}}
\caption{A figure with text\label{fig:withtext}}
\end{center}\end{figure}

\begin{lstlisting}[language=C++, caption={Cpp Testing}]
#include <iostream>
\end{lstlisting}

% Note to self, refer to all go programs are go programs, golang is reserved for the code.
\begin{lstlisting}[caption={Golang Program}]
package main

import (
"fmt"
"io/ioutil"
)
\end{lstlisting}


\lstinputlisting[language=Octave, caption=Python Script]{scripts/code.py}

Using references is \cite{book:2300108}
\section{Some math}

Inline math: $r = r_0 + vt - \frac{1}{2}at^2$
followed by display math:
\begin{equation}
a^2 + b^2 = c^2
\end{equation}




\chapter{Real Content}

\begin{table}
\begin{tabular}{p{3cm} c c c c}Name & Category & Priority \\ \hline  examin pyalgotrade for stock selling and buying &  low &  investing \\ \hline  work on web scrapper experiment felgo &  webscrap &  high \\ \hline  gas station network  &  high &  dapps \\ \hline  explore ipfs solutions such as pinata and textile &   &  \\ \hline  add dash auth to dashboard &  finance &  high \\ \hline
\end{tabular}
\caption{\textbf{Todo List 2019/8/5}}
\end{table}

Enable Google Cloud Build, need to update google cloud sdk on windows, might broken.

Also it seems that google cloud build.




\begin{mdframed}[style=important, frametitle={QPU Versus GPU: Some Common Characteristics}]

\begin{itemize}
\item It is very rare that a program will run entirely on a QPU. Usually, a program running on a CPU will issue QPU instructions, and later retrieve the results.
\item Some tasks are very well suited to the QPU, and others are not.
\item The QPU runs on a separate clock from the CPU, and usually has its own dedicated hardware interfaces to external devices (such as optical outputs).
\item A typical QPU has its own special RAM, which the CPU cannot efficiently access.
\item Here's a list of pertinent facts about what it’s like to program a QPU:
It is very rare that a program will run entirely on a QPU. Usually, a program running on a CPU will issue QPU instructions, and later retrieve the results.
Some tasks are very well suited to the QPU, and others are not.
The QPU runs on a separate clock from the CPU, and usually has its own dedicated hardware interfaces to external devices (such as optical outputs).
A typical QPU has its own special RAM, which the CPU cannot efficiently access.
\end{itemize}
\end{mdframed}

% Having java here, doesn't matter because the language gets mapped in the html anyway using prism
\begin{lstlisting}[language=java, caption=Javascript Program: Quantum random spy hunter]
Example 2-4. Quantum random spy hunter
qc.reset(3);
qc.discard();
var a = qint.new(1, 'alice');
var fiber = qint.new(1, 'fiber');
var b = qint.new(1, 'bob');

function random_bit(q) {
    q.write(0);
    q.had();
    return q.read();
}

// Generate two random bits
var send_had = random_bit(a);
var send_val = random_bit(a);

// Prepare Alice's qubit
a.write(0);
if (send_val)  // Use a random bit to set the value
    a.not();
if (send_had)  // Use a random bit to apply HAD or not
    a.had();

// Send the qubit!
fiber.exchange(a);

// Activate the spy
var spy_is_present = true;
\end{lstlisting}

\textbf{Magneta Music Generation} Some of the beats generated were acceptable, could make sound effects from randomized shorter effects.

Probably just trial and error as for images, swipe them off the internet, because I still need animations for now.


\section{SQL}

Query
\begin{lstlisting}[caption={SQL Query for length}]
SELECT first_name, length(first_name) as first_length, last_name, length(last_name) as last_length, length(first_name)+length(last_name) as total_length FROM users WHERE length(first_name)+length(last_name) > 30;
\end{lstlisting}

\begin{lstlisting}[caption={SQL Query for addresses}]
SELECT DISTINCT ON (a.place_id)
            a.full_json #>> '{5,short_name}' AS state
          , a.full_json #>> '{7,long_name}'  AS postal_code
          , a.full_json #>> '{3,short_name}' AS city
          FROM   addresses_autocomplete a
          JOIN   regions r ON r.short_name = a.full_json #>> '{5,short_name}'
          WHERE  a.formatted = $1
          AND    json_array_length(a.full_json) > 7
          AND    r.country_id = 1
\end{lstlisting}

\subsection{Publishing Dart Packages With Docker}

In order to publish a package in dart you need to authenticate with a web browser. This is easily done if dart is installed locally, however if you are using dart in docker this becomes a bit more confusing.
After authentication with google, copy the response url and curl it within the docker container.

`pub publish'

Go to url in web browser, copy url directed at `localhost` and curl inside docker container.

I believe pub publish spins out a http listener at a localhost port which is why using curl to hit it works in dart.

% Main file dump location

Learned that `::1` means localhost for ip 6, and express has a useful function called req.ip to get client ip, but fails if you are using a proxy.

\begin{lstlisting}[caption={Javascript Program get ip address}]
var ip = req.headers['x-forwarded-for'] || 
     req.connection.remoteAddress || 
     req.socket.remoteAddress ||
     (req.connection.socket ? req.connection.socket.remoteAddress : null);
\end{lstlisting}

\begin{lstlisting}[caption={Javascript Program HTTP server}]
var http = require("http"),
    url = require("url"),
    path = require("path"),
    fs = require("fs"),
    port = process.argv[2] || 8888;

http.createServer(function(request, response) {

  var uri = url.parse(request.url).pathname
    , filename = path.join(process.cwd(), uri);

  fs.exists(filename, function(exists) {
    if(!exists) {
      response.writeHead(404, {"Content-Type": "text/plain"});
      response.write("404 Not Found\n");
      response.end();
      return;
    }

    if (fs.statSync(filename).isDirectory()) filename += '/index.html';

    fs.readFile(filename, "binary", function(err, file) {
      if(err) {        
        response.writeHead(500, {"Content-Type": "text/plain"});
        response.write(err + "\n");
        response.end();
        return;
      }

      response.writeHead(200);
      response.write(file, "binary");
      response.end();
    });
  });
}).listen(parseInt(port, 10));
\end{lstlisting}

\begin{lstlisting}[language=C++, caption={Cpp Testing}]
#include <iostream>
\end{lstlisting}



\paragraph{Thoughts}

Setting on felgo cloud builds was a pain, but somehow I managed to do it.

Need to tag that particular release, the problem was a mix of license code, the way numbering works for felgo among other things.

The issue was plugins I did not need, but were added in.

Looking into admob for google suggests I need around 2k view/downloads to make money,

Will shove it in hopefully for something to happen.

Kinda lazy with the felgo games.


\begin{mdframed}[style=theorem, frametitle={Algorithm Features}]
	A good algorithm must have three features: correctness, maintainability, and
effciency.
\end{mdframed}

\begin{mdframed}[style=important, frametitle={Data Structures}]
The term data structure refers to the organization of data in a
computer's memory, in order to retrieve it quickly for processing.
It is a scheme for data organization to decouple the functional
definition of a data structure from its implementation. A data
structure is chosen based on the problem type and the
operations performed on the data.
\end{mdframed}

There are five basic rules for calculating an algorithm’s Big O notation.

\begin{enumerate}
\item If an algorithm performs a certain sequence of steps f(N) times for a mathematical function f, then it takes O(f(N)) steps.
\item  If an algorithm performs an operation that takes O(f(N)) steps and then
performs a second operation that takes O(g(N)) steps for functions f and
g, then the algorithm’s total performance is O(f(N) g(N)).
\item  If an algorithm takes $O(f(N)+g(N))$ time and the function f(N) is greater
than g(N) for large N, then the algorithm’s performance can be simplified
to $O(f(N))$.
\item  If an algorithm performs an operation that takes $O(f(N))$ steps, and for
every step in that operation it performs another $O(g(N))$ steps, then the
algorithm’s total performance is $O(f(N)\times g(N)$.
\item  Ignore constant multiples. If C is a constant, $O(C \times f(N))$ is the same as
$O(f(N))$, and $O(C \times f(N))$ is the same as O(f(N)).

Adapter, bridge, composite,
decorator, facade, flyweight, private class data, and proxy are the
Gang of Four (GoF) structural design patterns. 

The adapter pattern comprises the target, adaptee, adapter, and
client:
\begin{itemize}
\item Target is the interface that the client calls and invokes
methods on the adapter and adaptee.
\item The client wants the incompatible interface implemented
by the adapter.
\item The adapter translates the incompatible interface of the
adaptee into an interface that the client wants.
\end{itemize}

\lstinputlisting[caption={Golang Adaptor Example}]{chapters/important/algorithms/adaptor.go}

\textbf{Bridge}
Bridge decouples the implementation from the abstraction. The
abstract base class can be subclassed to provide different
implementations and allow implementation details to be
modified easily. The interface, which is a bridge, helps in making
the functionality of concrete classes independent from the
interface implementer classes. The bridge patterns allow the
implementation details to change at runtime.

\lstinputlisting[caption={Golang Adaptor Example}]{chapters/important/algorithms/bridge.go}


\begin{mdframed}[style=important, frametitle={Composite}]
A composite is a group of similar objects in a single object.
Objects are stored in a tree form to persist the whole hierarchy.
The composite pattern is used to change a hierarchical collection
of objects. 
\end{mdframed}

The composite pattern comprises the component interface, component
class, composite, and client:
\begin{itemize}
	\item The component interface defines the default behavior of all
	objects and behaviors for accessing the components of
	the composite.
	\item  The composite and component classes implement the component
	interface.
	\item The client interacts with the component interface to
	invoke methods in the composite.
\end{itemize}

\begin{mdframed}[style=important, frametitle={Decorator}]
In a scenario where class responsibilities are removed or added,
the decorator pattern is applied. The decorator pattern helps
with subclassing when modifying functionality, instead of static
inheritance. An object can have multiple decorators and runtime decorators. The single responsibility principle can be
achieved using a decorator.
\end{mdframed}

% Note to self, refer to all go programs are go programs, golang is reserved for the code.
\begin{lstlisting}[caption={Golang Program}]
package main

import (
	"fmt"
	"io/ioutil"
)
\end{lstlisting}


\begin{warpprint}   % For print output ...
\cleardoublepage    % ... a common method to place index entry into TOC.
\phantomsection
\addcontentsline{toc}{chapter}{\indexname}
\end{warpprint}
\ForceHTMLPage      % HTML index will be on its own page.
\ForceHTMLTOC       % HTML index will have its own toc entry.
\printindex

\nocite{*}
\printbibliography

\glsaddall
\printglossaries

\end{document}