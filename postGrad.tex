
% Save this as tutorial.tex for the lwarp package tutorial.

% Listings works with book
\documentclass{book}

\usepackage{iftex}

% --- LOAD FONT SELECTION AND ENCODING BEFORE LOADING LWARP ---

\ifPDFTeX
\usepackage{lmodern}            % pdflatex or dvi latex
\usepackage[T1]{fontenc}
\usepackage[utf8]{inputenc}
\else
\usepackage{fontspec}           % XeLaTeX or LuaLaTeX
\fi

% --- LWARP IS LOADED NEXT ---
\usepackage[
HomeHTMLFilename=index,     % Filename of the homepage.
%   HTMLFilename={node-},       % Filename prefix of other pages.
%   IndexLanguage=english,      % Language for xindy index, glossary.
%   latexmk,                    % Use latexmk to compile.
%   OSWindows,                  % Force Windows. (Usually automatic.)
% Mathjax is the way to go, if I lose mathjax, host it myself????? Don't have too much math involved in my equations.
mathjax,                    % Use MathJax to display math.
]{lwarp}
% \boolfalse{FileSectionNames}  % If false, numbers the files.


% --- Custom non tutorial content
% --- Listings
\usepackage{listings}

% Styling only for pdf --- for now
\lstset{language=C++,
	basicstyle=\ttfamily,
	keywordstyle=\color{blue}\ttfamily,
	stringstyle=\color{red}\ttfamily,
	commentstyle=\color{green}\ttfamily,
	morecomment=[l][\color{magenta}]{\#},
	captionpos=t,
}

% --- Glossary Terms
\usepackage[acronym, toc, xindy]{glossaries}
% Hands on Systems Programming C++
\newglossaryentry{api}
{
   name={API},
   description={An Application Programming Interface (API) is a particular set
           of rules and specifications that a software program can follow to access and make use of the services and resources provided by another particular software program that implements that API},
   first={Application Programming Interface (API)},
   long={Application Programming Interface}
}

\newglossaryentry{staticlib}
{
	name={Static libraries},
	description ={
			libraries that are linked at compile time
	}
}

\newglossaryentry{dynamiclib}
{
	name={Dynamic libraries},
	description ={
			libraries that are linked at run time
	}
}

%%% Laravel Notes

\newglossaryentry{depInj}
{
	name={Dependency Injection},
	description={
		Dependency injection means that, rather than being instantiated ("newed up") within a class, each class's dependencies will be injected in from the outside.
	}
}

\newglossaryentry{IoC}
{
	name={IoC},
	description={
		In software engineering, inversion of control (IoC) is a programming principle. IoC inverts the flow control as compared to traditional control flow. In IoC, custom-written portions of a computer program receive the flow of control from a generic framework. A software architecture with this design inverts control as compared to traditional procedural programming: in traditional programming, the custom code that expresses the purpose of the program calls into reusable libraries to take care of generic tasks, but with inversion of control, it is the framework that calls into the custom, or task-specific, code.
	},
	first={Inversion of Control},
	long={Inversion of Control}
}

%% --- GAN Notes ---
\newglossaryentry{GAN}
{
	name={GAN},
	description={
		A GAN is a deep neural network architecture made up of two networks, a generator network and a discriminator network. Through multiple cycles of generation and discrimination, both networks train each other, while simultaneously trying to outwit each other.
	},
	first={Generative Adversarial Networks},
	long={Generative Adversarial Networks}
}
% --- Software Tools ---
\newglossaryentry{git}
{
	name={git},
	description ={
			the most popular version control system used for tracking changes in source code and coordinating work on those files among multiple people
	}
}

% --- Quantum Computing ---


 % Glossary File
\makeglossaries

% Hands on Systems Programming C++
\newacronym{api}{API}{application programming interface}

\newacronym{abi}{ABI}{application binary interface}

\newacronym{qpu}{QPU}{Quantum Processing Unit}
\glsaddall


% --- MDFFRAMED CUSTOMIZATION
\usepackage{mdframed}

\mdfdefinestyle{example}{
	linecolor=blue!50,
	align=center, 
	backgroundcolor=blue!30, 
	frametitlebackgroundcolor=blue!15,
	shadow=true
}
\mdfdefinestyle{definition}{
	linecolor=red!50,
	align=center, 
	backgroundcolor=red!30, 
	frametitlebackgroundcolor=red!15,
	shadow=true
}
\mdfdefinestyle{theorem}{
	linecolor=yellow!50,
	align=center, 
	backgroundcolor=yellow!30, 
	frametitlebackgroundcolor=yellow!15,
	shadow=true
}

\mdfdefinestyle{important}{
	linecolor=purple!50,
	align=center, 
	backgroundcolor=purple!30, 
	frametitlebackgroundcolor=purple!15,
	shadow=true
}

% --- References ---
\newcommand{\onlineCite}{[Online] Available: }	% Used in BIBLIOGRAPHY 
\usepackage[backend=bibtex,sorting=none]{biblatex}	% Sort by citation order, IEEE
\addbibresource{config/references.bib}

% --- MATH LIBRARIES ---
\usepackage{amssymb}
% --- LOAD PDFLATEX MATH FONTS HERE ---

% --- OTHER PACKAGES ARE LOADED AFTER LWARP ---
\usepackage{makeidx} \makeindex
\usepackage{xcolor}             % (Demonstration purposes only.)
\usepackage{hyperref,cleveref}  % LOAD THESE LAST!

% --- LATEX AND HTML CUSTOMIZATION ---
\title{The Lwarp Tutorial}
\author{Some Author}
\setcounter{tocdepth}{2}        % Include subsections in the \TOC.
\setcounter{secnumdepth}{2}     % Number down to subsections.
\setcounter{FileDepth}{1}       % Split \HTML\ files at sections
\booltrue{CombineHigherDepths}  % Combine parts/chapters/sections
\setcounter{SideTOCDepth}{1}    % Include subsections in the side\TOC
\HTMLTitle{Webpage Title}       % Overrides \title for the web page.
\HTMLAuthor{Some Author}        % Sets the HTML meta author tag.
\HTMLLanguage{en-US}            % Sets the HTML meta language.
\HTMLDescription{A description.}% Sets the HTML meta description.
\HTMLFirstPageTop{Name and \fbox{HOMEPAGE LOGO}}
\HTMLPageTop{\fbox{LOGO}}
\HTMLPageBottom{Contact Information and Copyright}
\CSSFilename{lwarp_sagebrush.css}


%%%%%%%%%%%%%%%%%%%%%%%%%%%%%%%%%%%%%%%%%%%%%%%%%%%%%%%%%%%%%%%%%%%%%%%%%%%%%%%%
%%%%%%%%%%%%%%%%%%%%%%%% UTILITY COMMANDS %%%%%%%%%%%%%%%%%%%%%%%%%%%%%%%%%%%%%%
%%%%%%%%%%%%%%%%%%%%%%%%%%%%%%%%%%%%%%%%%%%%%%%%%%%%%%%%%%%%%%%%%%%%%%%%%%%%%%%%

%%%%%%%%%%%%%%%%%%%%%%%%%%%%%%%%%%%%%%%%%%%%%%%%%%%%%%%%%%%%%%%%%%%%%%%%%%%%%%%%%
%%%%%%%%%%%%%%%%%%%%%%%%%%%%%% REFERENCE COMMANDS OF INTEREST %%%%%%%%%%%%%%%%%%%
%%%%%%%%%%%%%%%%%%%%%%%%%%%%%%%%%%%%%%%%%%%%%%%%%%%%%%%%%%%%%%%%%%%%%%%%%%%%%%%%%

%  replacementTerms = ["Cpp Code", cplusplus, "Latex Code", latex, "Python Script", python, 
% "Bash Script", bash,"Matlab Script", matlab,"Yaml File",yaml,"JSON Output", json, "Golang", golang]
% These are the terms that are manually converted to code listings, by default it is converted to cpp, thanks to SENG475

% Other Notes

% --- Mdframed examples
% 	Values include:	example, definition, theorem and important
%
%	\begin{mdframed}[style=definition, frametitle={SENG475 Example Code Listing}]
%		All for the sake of glory city. We see that in \cref{ex:1} that mdframed makes 
%		environments look nice
%	\end{mdframed}
%	
%	\begin{mdframed}[style=theorem, frametitle={SENG475 Example Code Listing}]
%		All for the sake of glory city. We see that in \cref{ex:1} that mdframed makes 
%		environments look nice
%	\end{mdframed}
%	
%	\begin{mdframed}[style=important, frametitle={SENG475 Example Code Listing}]
%		All for the sake of glory city. We see that in \cref{ex:1} that mdframed makes 
%		environments look nice
%	\end{mdframed}
\begin{document}


\begin{warpHTML}
	\maketitle                   % Or titlepage/titlingpage environment.
\end{warpHTML}
\begin{center}
	%		{\vspace*{3pt} }
	{\Huge \textsc{Personal Study Notes} \\ \vspace{40pt}}
	%	{\Huge  \\ \vspace{4pt}}  
	\rule[13pt]{1\textwidth}{1pt} \\ \vspace{1pt}
	{\LARGE \textbf{{\textsc{Software Engineer}}} \\ }
	{\Large \textsc{Post Grad Notes} \\} 
	\vspace{4pt} 
	%{\Large \textsc{Spring 2018}} \\ 
	\vspace{20pt}
	{\Large \textsc{Victoria, British Columbia, Canada} \\ \vspace{45pt} }
	
	%		\\ \vspace{10pt}
	{\Large \textsc{\today} \\ \vspace{15pt}
		\textbf{Name:} \hfill David Li \\
		\textbf{Student Number:} \hfill V00818631 \\
		\textbf{Email:} \hfill \href{mailto:davidli012345@gmail.ca}{davidli012345@gmail.ca} \\
		\vspace{15pt}
		{\Large \textsc{In partial fulfillment of the never ending quest to learn. \\
			}
		}	
	}
\end{center}

% An article abstract would go here.

\tableofcontents                % MUST BE BEFORE THE FIRST SECTION BREAK!
\listoffigures
\lstlistoflistings % THIS SEEMS TO WORK WITH THE BOOK CLASS

\chapter{Example chapter}

\section{A section}

This is some text which is indexed.\index{Some text.}

\subsection{A subsection}

See \cref{fig:withtext}.

\begin{mdframed}[style=definition, frametitle={SENG475 Example Code Listing}]
	All for the sake of glory city. We see that in \cref{ex:1} that mdframed makes 
	environments look nice
\end{mdframed}

\begin{mdframed}[style=theorem, frametitle={SENG475 Example Code Listing}]
	All for the sake of glory city. We see that in \cref{ex:1} that mdframed makes 
	environments look nice
\end{mdframed}

\begin{mdframed}[style=important, frametitle={SENG475 Example Code Listing}]
	All for the sake of glory city. We see that in \cref{ex:1} that mdframed makes 
	environments look nice
\end{mdframed}
	
\begin{figure}\begin{center}
\fbox{\textcolor{blue!50!green}{Text in a figure.}}
\caption{A figure with text\label{fig:withtext}}
\end{center}\end{figure}

\begin{lstlisting}[language=C++, caption={Cpp Testing}]
#include <iostream>
\end{lstlisting}

% Note to self, refer to all go programs are go programs, golang is reserved for the code.
\begin{lstlisting}[caption={Golang Program}]
package main

import (
"fmt"
"io/ioutil"
)
\end{lstlisting}


\lstinputlisting[language=Octave, caption="Python Script"]{scripts/code.py}

Using references is \cite{book:2300108}
\section{Some math}

Inline math: $r = r_0 + vt - \frac{1}{2}at^2$
followed by display math:
\begin{equation}
a^2 + b^2 = c^2
\end{equation}




\chapter{Real Content}

\begin{table}
\begin{tabular}{p{3cm} c c c c}Name & Category & Priority \\ \hline  examin pyalgotrade for stock selling and buying &  low &  investing \\ \hline  work on web scrapper experiment felgo &  webscrap &  high \\ \hline  gas station network  &  high &  dapps \\ \hline  explore ipfs solutions such as pinata and textile &   &  \\ \hline  add dash auth to dashboard &  finance &  high \\ \hline
\end{tabular}
\caption{\textbf{Todo List 2019/8/5}}
\end{table}

Enable Google Cloud Build, need to update google cloud sdk on windows, might broken.

Also it seems that google cloud build.

% Ignore this




\begin{mdframed}[style=important, frametitle={QPU Versus GPU: Some Common Characteristics}]

\begin{itemize}
\item It is very rare that a program will run entirely on a QPU. Usually, a program running on a CPU will issue QPU instructions, and later retrieve the results.
\item Some tasks are very well suited to the QPU, and others are not.
\item The QPU runs on a separate clock from the CPU, and usually has its own dedicated hardware interfaces to external devices (such as optical outputs).
\item A typical QPU has its own special RAM, which the CPU cannot efficiently access.
\item Here's a list of pertinent facts about what it’s like to program a QPU:
It is very rare that a program will run entirely on a QPU. Usually, a program running on a CPU will issue QPU instructions, and later retrieve the results.
Some tasks are very well suited to the QPU, and others are not.
The QPU runs on a separate clock from the CPU, and usually has its own dedicated hardware interfaces to external devices (such as optical outputs).
A typical QPU has its own special RAM, which the CPU cannot efficiently access.
\end{itemize}
\end{mdframed}

% Having java here, doesn't matter because the language gets mapped in the html anyway using prism
\begin{lstlisting}[language=java, caption=Javascript Program: Quantum random spy hunter]
Example 2-4. Quantum random spy hunter
qc.reset(3);
qc.discard();
var a = qint.new(1, 'alice');
var fiber = qint.new(1, 'fiber');
var b = qint.new(1, 'bob');

function random_bit(q) {
    q.write(0);
    q.had();
    return q.read();
}

// Generate two random bits
var send_had = random_bit(a);
var send_val = random_bit(a);

// Prepare Alice's qubit
a.write(0);
if (send_val)  // Use a random bit to set the value
    a.not();
if (send_had)  // Use a random bit to apply HAD or not
    a.had();

// Send the qubit!
fiber.exchange(a);

// Activate the spy
var spy_is_present = true;
\end{lstlisting}
\begin{lstlisting}[language=C++, caption={Cpp Testing}]
#include <iostream>
\end{lstlisting}



\paragraph{Thoughts}

Setting on felgo cloud builds was a pain, but somehow I managed to do it.

Need to tag that particular release, the problem was a mix of license code, the way numbering works for felgo among other things.

The issue was plugins I did not need, but were added in.

Looking into admob for google suggests I need around 2k view/downloads to make money,

Will shove it in hopefully for something to happen.

\textbf{Magneta Music Generation} Some of the beats generated were acceptable, could make sound effects from randomized shorter effects.

Probably just trial and error as for images, swipe them off the internet, because I still need animations for now.


\section{SQL}

Query
\begin{lstlisting}[caption={SQL Query for length}]
SELECT first_name, length(first_name) as first_length, last_name, length(last_name) as last_length, length(first_name)+length(last_name) as total_length FROM users WHERE length(first_name)+length(last_name) > 30;
\end{lstlisting}

\begin{lstlisting}[caption={SQL Query for addresses}]
SELECT DISTINCT ON (a.place_id)
            a.full_json #>> '{5,short_name}' AS state
          , a.full_json #>> '{7,long_name}'  AS postal_code
          , a.full_json #>> '{3,short_name}' AS city
          FROM   addresses_autocomplete a
          JOIN   regions r ON r.short_name = a.full_json #>> '{5,short_name}'
          WHERE  a.formatted = $1
          AND    json_array_length(a.full_json) > 7
          AND    r.country_id = 1
\end{lstlisting}

\begin{mdframed}[style=theorem, frametitle={Algorithm Features}]
	A good algorithm must have three features: correctness, maintainability, and
effciency.
\end{mdframed}

There are five basic rules for calculating an algorithm’s Big O notation.

\begin{enumerate}
\item If an algorithm performs a certain sequence of steps f(N) times for a mathematical function f, then it takes O(f(N)) steps.
\item  If an algorithm performs an operation that takes O(f(N)) steps and then
performs a second operation that takes O(g(N)) steps for functions f and
g, then the algorithm’s total performance is O(f(N) g(N)).
\item  If an algorithm takes $O(f(N)+g(N))$ time and the function f(N) is greater
than g(N) for large N, then the algorithm’s performance can be simplified
to $O(f(N))$.
\item  If an algorithm performs an operation that takes $O(f(N))$ steps, and for
every step in that operation it performs another $O(g(N))$ steps, then the
algorithm’s total performance is $O(f(N)\times g(N)$.
\item  Ignore constant multiples. If C is a constant, $O(C \times f(N))$ is the same as
$O(f(N))$, and $O(C \times f(N))$ is the same as O(f(N)).

\end{enumerate}
% Note to self, refer to all go programs are go programs, golang is reserved for the code.
\begin{lstlisting}[caption={Golang Program}]
package main

import (
	"fmt"
	"io/ioutil"
)
\end{lstlisting}

\chapter{Example chapter}

\section{A section}

This is some text which is indexed.\index{Some text.}

\subsection{A subsection}

See \cref{fig:withtext}.

\begin{mdframed}[style=definition, frametitle={SENG475 Example Code Listing}]
	All for the sake of glory city. We see that in \cref{ex:1} that mdframed makes 
	environments look nice
\end{mdframed}

\begin{mdframed}[style=theorem, frametitle={SENG475 Example Code Listing}]
	All for the sake of glory city. We see that in \cref{ex:1} that mdframed makes 
	environments look nice
\end{mdframed}

\begin{mdframed}[style=important, frametitle={SENG475 Example Code Listing}]
	All for the sake of glory city. We see that in \cref{ex:1} that mdframed makes 
	environments look nice
\end{mdframed}
	
\begin{figure}\begin{center}
\fbox{\textcolor{blue!50!green}{Text in a figure.}}
\caption{A figure with text\label{fig:withtext}}
\end{center}\end{figure}

\begin{lstlisting}[language=C++, caption={Cpp Testing}]
#include <iostream>
\end{lstlisting}

% Note to self, refer to all go programs are go programs, golang is reserved for the code.
\begin{lstlisting}[caption={Golang Program}]
package main

import (
"fmt"
"io/ioutil"
)
\end{lstlisting}


\lstinputlisting[language=Octave, caption="Python Script"]{scripts/code.py}

Using references is \cite{book:2300108}
\section{Some math}

Inline math: $r = r_0 + vt - \frac{1}{2}at^2$
followed by display math:
\begin{equation}
a^2 + b^2 = c^2
\end{equation}




\end{document}