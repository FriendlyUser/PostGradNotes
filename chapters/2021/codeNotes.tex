

\begin{lstlisting}
import { useState, useEffect } from 'react'
const useFetch = (initialUrl, initialParams = {}, skip = false) => {
  const [url, updateUrl] = useState(initialUrl)
  const [params, updateParams] = useState(initialParams)
  const [data, setData] = useState(null)
  const [isLoading, setIsLoading] = useState(false)
  const [hasError, setHasError] = useState(false)
  const [errorMessage, setErrorMessage] = useState('')
  const [refetchIndex, setRefetchIndex] = useState(0)
const queryString = Object.keys(params)
    .map((key) => encodeURIComponent(key) + '=' +
    encodeURIComponent(params[key])).join('&')
const refetch = () => setRefetchIndex((prevRefetchIndex) => prevRefetchIndex + 1)
useEffect(() => {
    const fetchData = async () => {
      if (skip) return
      setIsLoading(true)
      try {
        const response = await fetch(`${url}${queryString}`)
        const result = await response.json()
        if (response.ok) {
          setData(result)
        } else {
          setHasError(true)
          setErrorMessage(result)
        }
      } catch (err) {
        setHasError(true)
        setErrorMessage(err.message)
      } finally {
        setIsLoading(false)
      }
    }
    fetchData()
  }, [url, params, refetchIndex])
return { data, isLoading, hasError, errorMessage, updateUrl, updateParams, refetch }
}
export default useFetch
\end{lstlisting}

To generate a new migration on windows.

\begin{lstlisting}
npx ts-node node_modules/typeorm/cli.js migration:generate -n AddUserIdToNodeJobs
\end{lstlisting}

to run migrations manually for typeorm on windows, build javascript files

\begin{lstlisting}
npm run build
\end{lstlisting}

Point to the javascript files
\begin{lstlisting}
TYPEORM_MIGRATIONS=dist/src/database/migrationsTest/*.js
TYPEORM_MIGRATIONS_DIR=dist/src/database/migrationsTest
TYPEORM_ENTITIES=dist/src/api/models/*.js
TYPEORM_ENTITIES_DIR=dist/src/api/models
\end{lstlisting}

Use the migration commands with standard nodejs
\begin{lstlisting}
node node_modules/typeorm/cli.js migrations:show
\end{lstlisting}

\begin{lstlisting}[caption={SQL Query Count}]
SELECT status, count(*) as scan_status FROM js GROUP BY status
\end{lstlisting}

\begin{lstlisting}[caption={SQL Query example}]
SELECT AVG(number_of_star) AS meanRating FROM review WHERE vehicle_id = ?
\end{lstlisting}

\begin{lstlisting}[caption={SQL Query example verification}]
SELECT su.id AS userId, system_user.email, js.first_name AS firstName,
    js.last_name AS lastName, su.is_verified_driver_license AS systemApproved,
    js.id AS scanId, js.status AS scanStatus, js.created_date AS scanDate,
    id.id AS decisionId, id.decision AS decisionStatus,
    id.created_at AS decisionDate
    FROM su
    LEFT JOIN js ON su.id = js.user_id
    LEFT JOIN id ON su.id = id.user_id
    WHERE js.created_date BETWEEN ? AND ?
    ORDER BY js.created_date
\end{lstlisting}

\begin{lstlisting}[caption={SQL query to get entry from json data}]
SELECT id, job_type, run_date, data, version, user_id,
    JSON_EXTRACT(JSON_UNQUOTE(data), '$.tripId') as tripId
    FROM nj where
    JSON_EXTRACT(JSON_UNQUOTE(data), '$.tripId') = ?
    and (job_type = 'NEW')`;
\end{lstlisting}
